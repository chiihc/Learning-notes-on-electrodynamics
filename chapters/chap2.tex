\cleardoublepage

\section{静电学}
\subsection{电场}
\mysssec{库仑定律}
设一个静止点电荷$q_1$距检验电荷$q_2$的距离为$r$, 那么它作用在$Q$上的力是
\begin{equation}
    \boldsymbol{F}=\frac{1}{4\pi\varepsilon_0}\frac{q_1q_2}{r^3}\boldsymbol{r}
\end{equation}
常数$\varepsilon_0$称为真空介电常数, $\varepsilon_0=8.85\times10^{-12}\ \rm{\dfrac{C^2}{N\cdot m^2}}$
\mysssec{电场}
\begin{equation}
    \boldsymbol{E}\equiv\frac{\boldsymbol{F}}{Q}=\frac{1}{4\pi\varepsilon_0}\frac{q}{r^3}\boldsymbol{r}
\end{equation}
\mysssec{电场强度通量}
$\boldsymbol{\Phi }_E$为电场强度通量
\begin{equation}
    \boldsymbol{\Phi }_E\equiv\iint_S\boldsymbol{E}\cdot\dif \boldsymbol{S}
\end{equation}
\mysssec{高斯定理}
$\rho$为电荷密度
\begin{equation}
    \boldsymbol{\nabla}\cdot\boldsymbol{E}=\frac{1}{\varepsilon_0}\rho
\end{equation}
\mysssec{两个带电为$q$的电荷, 相距$d$放置, 求垂直于连线中点且距离为$x_2$处的电场, 如果把一个换成$-q$会怎样?当$y>>d$时会怎样?}
\begin{align*}
    \frac{1}{4\pi\varepsilon_0}\frac{q}{r^3}\boldsymbol r_1
    +\frac{1}{4\pi\varepsilon_0}\frac{q}{r^3}\boldsymbol r_2
    &=2\frac{1}{4\pi\varepsilon_0}\frac{q}{r^3}x_2\boldsymbol{e}_2\\
    &=\frac{1}{2\pi\varepsilon_0}\frac{qx_2}{r^3}\boldsymbol{e}_2\\
    &=\dfrac{1}{2\pi\varepsilon_0}\dfrac{qx_2}{(x_2^2+\dfrac{d^2}{4})^\frac{3}{2}}\boldsymbol{e}_2
\end{align*}
\begin{align*}
    \frac{1}{4\pi\varepsilon_0}\frac{q}{r^3}\boldsymbol r_1
    +\frac{1}{4\pi\varepsilon_0}\frac{q}{r^3}\boldsymbol r_2
    &=2\frac{1}{4\pi\varepsilon_0}\frac{q}{r^3}\frac{d}{2}\boldsymbol{e}_1\\
    &=\frac{1}{4\pi\varepsilon_0}\frac{qd}{r^3}\boldsymbol{e}_1\\
    &\approx\frac{1}{4\pi\varepsilon_0}\frac{qd}{x_2^3}\boldsymbol{e}_1
\end{align*}
\mysssec{一个长度为$2L$的细杆均匀带电, 电荷线密度为$\lambda $, 求垂直于杆且与杆中心距离为$x_2$处的电场}
\begin{align*}
    \int_{-L}^L\frac{1}{4\pi\varepsilon_0}\frac{\lambda}{r^3}\boldsymbol{r}\dif x
    &=\int_{-L}^L\frac{1}{4\pi\varepsilon_0}\frac{\lambda}{r^3}{x_2\boldsymbol e_2}\dif x\\&
    =\frac{\lambda}{4\pi\varepsilon_0}\boldsymbol e_2\int_{-L}^L\frac{x_2}{r^3}\dif x\\&
    =\frac{\lambda}{4\pi\varepsilon_0}\boldsymbol e_2\int_{-L}^L\frac{x_2}{(x^2+x_2^2)^\frac{3}{2}}\dif x\\&
    \xlongequal{x=x_2\tan\theta}\frac{\lambda}{4\pi\varepsilon_0}\boldsymbol e_2\int_{-\arctan \frac{L}{x_2}}^{\arctan \frac{L}{x_2}}\frac{x_2}{(x_2^2\tan\theta^2+x_2^2)^\frac{3}{2}}\dif x_2\tan\theta\\&
    =\frac{\lambda}{4\pi\varepsilon_0}\boldsymbol e_2\int_{-\arctan \frac{L}{x_2}}^{\arctan \frac{L}{x_2}}\frac{x_2^2\cos^3\theta}{x_2^3\cos^2\theta}\dif\theta\\&
    =\frac{\lambda}{4\pi\varepsilon_0}\boldsymbol e_2\int_{-\arctan \frac{L}{x_2}}^{\arctan \frac{L}{x_2}}\frac{\cos\theta}{x_2^2}\dif \theta\\&
    =\frac{\lambda}{4\pi\varepsilon_0}\boldsymbol e_2\left.\frac{\sin\theta}{x_2}\right|_{-\arctan \frac{L}{x_2}}^{\arctan \frac{L}{x_2}}\\&
    =\frac{\lambda}{2\pi\varepsilon_0x_2}\boldsymbol e_2\frac{L}{\sqrt{L^2+x_2^2}}
\end{align*}
\mysssec{一个长度为$L$的细杆均匀带电, 电荷线密度为$\lambda $, 求杆一端上方距杆$y$处的电场}
\begin{align*}
    \int_{0}^{L}\frac{\lambda \boldsymbol e_{y}}{4\pi\varepsilon_0 r^2}\dif x_2
    &=\int_{0}^{L}\frac{\lambda \boldsymbol e_{y}}{4\pi\varepsilon_0 (x_2+y)^2}\dif x_2\\&
    =-\left.\frac{\lambda \boldsymbol e_{y}}{4\pi\varepsilon_0 (x_2+y)}\right|_{0}^{L}\\&
    =\frac{\lambda \boldsymbol e_{y}}{4\pi\varepsilon_0 (L+y)}-\frac{\lambda \boldsymbol e_{y}}{4\pi\varepsilon_0 y}\\&
    =\frac{\lambda \boldsymbol e_{y}L}{4\pi\varepsilon_0 (L+y)y}
\end{align*}
\mysssec{一个边长为$2L$的正方形线框均匀带电, 电荷线密度为$\lambda $, 求线框中心上方距$x_3$处的电场}
\begin{align*}
    4\frac{\lambda}{2\pi\varepsilon_0\sqrt{L^2+x_3^2}}\boldsymbol e_3\frac{L}{2\sqrt{2L^2+x_3^2}}\frac{x_3}{x_3\sqrt{L^2+x_3^2}^2}
    =\frac{\lambda}{\pi\varepsilon_0\sqrt{2L^2+x_3^2}}\boldsymbol e_3\frac{2Lx_3}{{L^2+x_3^2}}
\end{align*}
\mysssec{一个半径为$r$的圆线框均匀带电, 电荷线密度为$\lambda $, 求线框中心上方距$x_3$处的电场}
\begin{align*}
    2\pi r\frac{\lambda}{4\pi\varepsilon_0}\boldsymbol e_3\frac{x_3}{\sqrt{x_3^2+r^2}}
    =\frac{\lambda}{2\varepsilon_0}\boldsymbol e_3\frac{x_3r}{\sqrt{x_3^2+r^2}^3}
\end{align*}
\mysssec{一个半径为$r$的圆片均匀带电, 电荷面密度为$\sigma  $, 求圆片中心上方距$x_3$处的电场}
\begin{align*}
    \int_0^r\frac{\sigma}{2\varepsilon_0}\boldsymbol e_3\frac{x_3x_1}{\sqrt{x_3^2+x_1^2}^3}\dif x_1&
    =-\left.\frac{\sigma }{2\varepsilon_0}\boldsymbol e_3\frac{x_3}{\sqrt{x_3^2+x_1^2}}\right|_0^r\\&
    =\frac{\sigma }{2\varepsilon_0}\boldsymbol e_3\frac{x_3}{\sqrt{x_3^2}}
    -\frac{\sigma }{2\varepsilon_0}\boldsymbol e_3\frac{x_3}{\sqrt{x_3^2+r^2}}\\&
    =\frac{\sigma }{2\varepsilon_0}\boldsymbol e_3
    -\frac{\sigma }{2\varepsilon_0}\boldsymbol e_3\frac{x_3}{\sqrt{x_3^2+r^2}}
\end{align*}
\mysssec{一个半径为$r$的球面均匀带电, 电荷面密度为$\sigma  $, 求圆球上方距$d$处的电场(分$d>r$和$d<r$讨论)}
\begin{align*}
    \int_{\pi}^{0}\frac{\sigma}{2\varepsilon_0}\boldsymbol e_3\frac{(x_3+d)\sqrt{x_1^2+x_2^2}}{\sqrt{(x_3+d)^2+x_1^2+x_2^2}^3}\dif r\phi
    &=\int_{\pi}^{0}\frac{\sigma}{2\varepsilon_0}\boldsymbol e_3\frac{(r\cos\phi+d)\sqrt{x_1^2+x_2^2}}{\sqrt{(r\cos\phi+d)^2+x_1^2+x_2^2}^3}\dif r\phi\\
    &=\int_{\pi}^{0}\frac{\sigma}{2\varepsilon_0}\boldsymbol e_3\frac{(r\cos\phi+d)r\sin\phi}{\sqrt{r^2+2rd\cos\phi+d^2}^3}\dif r\phi\\
    &=\int_{0}^{\pi}\frac{\sigma}{2\varepsilon_0}\boldsymbol e_3\frac{(r\cos\phi+d)r^2\sin\phi}{\sqrt{r^2+2rd\cos\phi+d^2}^3}\dif \phi
    % &\xlongequal{u=r^2+2rd\cos\phi+d^2}\int_{0}^{\pi}\frac{\sigma r}{4\varepsilon_0d}\boldsymbol e_3\frac{d-r\cos\phi}{\sqrt{u}^3}\dif u\\
    % &=\int_{0}^{\pi}\frac{\sigma r}{2\varepsilon_0}\boldsymbol e_3\frac{u^2+d^2-r^2}{\sqrt{u}^3}\dif u
\end{align*}
\mysssec{在某个区域电场可以写为$\boldsymbol{E}=kr^3\boldsymbol{e_r}$, 求电荷密度和包含在半径为$R$, 球心在原点的闭合球面内的总电荷}
\begin{align*}
    \rho=\varepsilon_0\boldsymbol{\nabla}\cdot\boldsymbol{E}
    =\varepsilon_0\frac{1}{sr}\dfrac{\partial srE_r}{\partial r}
    =\varepsilon_0\frac{1}{sr}\dfrac{\partial srr^3}{\partial r}
    =\varepsilon_0\frac{1}{sr}{5sr^4}
    =5\varepsilon_0{r^3}k
\end{align*}
\begin{align*}
    4\pi\int_{0}^{R}5\varepsilon_0{r^3}\dif r=5\pi\varepsilon_0{R^4}k
\end{align*}
\subsection{电势}
\mysssec{定义}
\begin{equation}
    U(\boldsymbol{r})\equiv-\int_{O}^{r}\boldsymbol{E}\cdot\dif\boldsymbol{l}
\end{equation}
其中$O$为预先设置的标准参考点, 通常为无限远处
\mysssec{一个半径为$R$的均匀带电球体, 总电荷为$q$, 求电势$U(\boldsymbol{r})$}
$r\ge R$:
\begin{align*}
    \int_{r}^{\infty}\dfrac{q }{4\pi\varepsilon_0x^2}\dif x=
    \left.-\dfrac{q }{4\pi\varepsilon_0x}\right|_{r}^{\infty}
    =\dfrac{q }{4\pi\varepsilon_0r}
\end{align*}

$r<R$:
\begin{align*}
    \dfrac{q }{4\pi\varepsilon_0R}+\int_{r}^{R}\dfrac{q }{4\pi\varepsilon_0x^2}\frac{x^3}{R^3}\dif x
    &=\dfrac{q }{4\pi\varepsilon_0R}+\left.\dfrac{qx^2}{4\pi\varepsilon_0R^3}\right|_{r}^{R}\\
    &=\dfrac{q }{4\pi\varepsilon_0R}
    +\dfrac{qR^2}{4\pi\varepsilon_0R^3}
    -\dfrac{qr^2}{4\pi\varepsilon_0R^3}\\
    &=\dfrac{q }{2\pi\varepsilon_0R}-\dfrac{qr^2}{4\pi\varepsilon_0R^3}
\end{align*}
\mysssec{一条均匀带电的无限长直线, 电荷线密度为$\lambda$, 求电势$U(\boldsymbol{r})$}
\begin{align*}
    U=\int_r^\infty\frac{\lambda}{2\pi r}\dif x=
    \left.\frac{\lambda}{2\pi}\ln(x)\right|_r^\infty
    =\frac{1}{2\pi}\ln(r)
\end{align*}
\mysssec{一个长度为$2L$的细杆均匀带电, 电荷线密度为$\lambda $, 求垂直于杆且与杆中心距离为$h$处的电势}
\begin{align*}
    U&=\int_{h}^{\infty}\frac{\lambda}{4\pi\varepsilon_0}\frac{2L}{x_2\sqrt{L^2+x_2^2}}\dif x_2\\
    &=-\left.\frac{\lambda}{2\pi\varepsilon_0}\ln\left(\frac{\sqrt{x^2+L^2}+L}{x}\right)\right|_{h}^{\infty}
    \\
&=\frac{\lambda}{2\pi\varepsilon_0}\ln\left(\frac{\sqrt{h^2+L^2}+L}{h}\right)
\end{align*}
\mysssec{一个半径为$r$的圆片均匀带电, 电荷面密度为$\sigma  $, 求圆片中心上方距$h$处的电势}
\begin{align*}
    U&=\int_{h}^{\infty}\left(\frac{\sigma }{2\varepsilon_0}
    -\frac{\sigma }{2\varepsilon_0}\frac{x_3}{\sqrt{x_3^2+r^2}}\right)\dif x_3\\
    &=\left.\frac{\sigma }{2\varepsilon_0}x_3
    -\frac{\sigma }{2\varepsilon_0}{\sqrt{x_3^2+r^2}}\right|_{h}^{\infty}\\
    &=-\left.\frac{\sigma }{2\varepsilon_0}\frac{r^2}{x_3+\sqrt{x_3^2+r^2}}\right|_{h}^{\infty}\\
    &=\frac{\sigma }{2\varepsilon_0}\frac{r^2}{h+\sqrt{h^2+r^2}}
\end{align*}
\mysssec{一个半径为$r$的圆线框均匀带电, 电荷线密度为$\lambda $, 求线框中心上方距$h$处的电势}
\begin{align*}
    U=\int_{h}^{\infty}\frac{\lambda}{2\varepsilon_0}\frac{x_3r}{\sqrt{x_3^2+r^2}^3}\dif x_3
    =\left.-\frac{\lambda r}{2\varepsilon_0}\frac{1}{\sqrt{x_3^2+r^2}}\right|_{h}^{\infty}
    =\frac{\lambda r}{2\varepsilon_0}\frac{1}{\sqrt{h^2+r^2}}
\end{align*}
\mysssec{一个尖角向下的圆锥体均匀带电, 电荷面密度为$\sigma  $, 圆锥高度等于半径为$h$, 求$z=h$处的电势}
\begin{align*}
    U&=\int_{-h}^{0}\frac{\lambda (h+x_3)}{2\varepsilon_0}\frac{1}{\sqrt{(h-x_3)^2+(h+x_3)^2}}\dif x_3\\
    &=\int_{-h}^{0}\frac{\lambda (h+x_3)}{2\varepsilon_0}\frac{1}{\sqrt{2h^2+2x_3^2}}\dif x_3\\
    &=\frac{\lambda }{2\sqrt{2}\varepsilon_0}\int_{-h}^{0}\frac{h}{\sqrt{h^2+x_3^2}}+\frac{x_3}{\sqrt{h^2+x_3^2}}\dif x_3\\
    &=\left.h\ln\left({x_3}+{\sqrt{h^2+x_3^2}}\right)+\sqrt{h^2+x_3^2}\right|_{-h}^{0}\frac{\lambda }{2\sqrt{2}\varepsilon_0}\\
    &=\left[h\ln\left({\sqrt{h^2}}\right)
    +\sqrt{h^2}
    -h\ln\left({-h}+{\sqrt{2h^2}}\right)
    -\sqrt{2h^2}\right]\frac{\lambda }{2\sqrt{2}\varepsilon_0}\\
    &=\left[h\ln\left(\frac{1}{\sqrt{2}-1}\right)
    +h
    -\sqrt{2}h\right]\frac{\lambda }{2\sqrt{2}\varepsilon_0}\\
    &=\left[1
    -\sqrt{2}
    -\ln\left({\sqrt{2}-1}\right)\right]\frac{\lambda h}{2\sqrt{2}\varepsilon_0}
\end{align*}
\mysssec{一个半径为$r$的圆柱均匀带电, 电荷体密度为$\rho  $, 求圆柱中心上方距$h$处的电势}
\begin{align*}
    U&=\int_{-h}^{0}\frac{\rho}{2\varepsilon_0}\frac{r^2}{h+x_3+\sqrt{(h+x_3)^2+r^2}}\dif x_3\\
    &\xlongequal{u=h+x_3}\int_{0}^{h}\frac{\rho}{2\varepsilon_0}\frac{r^2}{u+\sqrt{u^2+r^2}}\dif u\\
    &=\int_{0}^{h}\frac{\rho r^2}{2\varepsilon_0}\frac{\sqrt{u^2+r^2}-u}{r^2}\dif u\\
    &=\frac{\rho}{2\varepsilon_0}\int_{0}^{h}{\sqrt{u^2+r^2}-u}\dif u\\
    &=\left.{u\sqrt{u^2+r^2}
    +r^2\ln\left(u+\sqrt{u^2+r^2}\right)
    -u^2}\right|_{0}^{h}\frac{\rho}{4\varepsilon_0}\\
    &=\left[{h\sqrt{h^2+r^2}
    +r^2\ln\left(h+\sqrt{h^2+r^2}\right)
    -h^2}
    -r^2\ln {r}\right]\frac{\rho}{4\varepsilon_0}
\end{align*}
\subsection{静电场的能量}
\mysssec{离散电荷体系的静电能}

考虑由 $N$ 个点电荷 $q_i$ 构成的静电体系。将电荷从无穷远逐个缓慢搬运到其最终位置, 
外力所做的总功即为体系的静电能。

设在放置第 $i$ 个电荷时, 其余电荷已就位, 则该电荷所处位置的当前电势为
\[
\varphi_i
=
\sum_{j=1}^{i-1}
\frac{1}{4\pi\varepsilon_0}
\frac{q_j}{|\boldsymbol r_i-\boldsymbol r_j|}.
\]

因此, 体系的总静电能为
\[
U
=
\sum_{i=1}^{N}\sum_{j=1}^{i-1}
\frac{1}{4\pi\varepsilon_0}
\frac{q_i q_j}{|\boldsymbol r_i-\boldsymbol r_j|}.
\]

注意到相互作用能在指标交换下满足
$U_{ij}=U_{ji}$, 
而对所有 $i\neq j$ 的求和中每一对指标被计数两次, 故可将上式对称化为
\[
U
=
\frac12
\sum_{\substack{i,j=1\\ i\neq j}}^{N}
\frac{1}{4\pi\varepsilon_0}
\frac{q_i q_j}{|\boldsymbol r_i-\boldsymbol r_j|}
=
\frac12 q_i \varphi_i.
\]


\mysssec{连续电荷分布的能量表达式}
对离散情形的自然推广给出
\[
U
=
\frac{1}{2}\iiint \rho\varphi\dif\tau.
\]

\mysssec{场能量}

静电势满足泊松方程
\[
\nabla^2 \varphi = -\frac{\rho}{\varepsilon_0}.
\]

将其代入能量表达式, 得
\[
U
=
-\frac{\varepsilon_0}{2}\iiint \varphi\,\nabla^2\varphi\,\dif\tau.
\]

对右端积分作分部积分。注意到
\[
\nabla\cdot(\varphi\nabla\varphi)
=
(\nabla\varphi)^2 + \varphi\nabla^2\varphi,
\]
于是
\[
\varphi\nabla^2\varphi
=
\nabla\cdot(\varphi\nabla\varphi)
-
(\nabla\varphi)^2.
\]

代入得
\[
U
=
-\frac{\varepsilon_0}{2}
\left[
\iiint \nabla\cdot(\varphi\nabla\varphi)\,\dif\tau
-
\iiint (\nabla\varphi)^2\,\dif\tau
\right].
\]

当 $r\to\infty$ 时, 
\[
\varphi(\boldsymbol r)\sim \frac{1}{r},\qquad
\nabla\varphi\sim \frac{1}{r^2}.
\]

因此
\[
\varphi\nabla\varphi \sim \frac{1}{r^3},
\]
对应的无穷远处曲面积分
\[
\oiint \varphi\nabla\varphi\cdot\dif\boldsymbol S\to0
\]

\[
U
=
\frac{\varepsilon_0}{2}\iiint (\nabla\varphi)^2\,\dif\tau.
\]

\[
U=\frac{\varepsilon_0}{2}\iiint E^2\,\dif\tau
\]
\mysssec{电场能量密度}
这表明:静电能可以视为分布在空间中的电场所携带的能量。

由此自然引入电场的能量密度
\[
\boxed{
u
\equiv
\frac{\varepsilon_0}{2}\boldsymbol E^2
}
\]
\mysssec{考虑两个同心球面, 半径分别为$a$和$b$, 内球面带有电荷$q$ , 外球面带有电荷$-q$, 求总能量}
\begin{align*}
    U&=\frac{\varepsilon_0}{2}\iiint E^2\dif\tau
    =\frac{\varepsilon_04\pi}{2}\int_a^b \left(\frac{q}{4\pi\varepsilon_0r^2}\right)^2r^2\dif r
    =\int_a^b \frac{q^2}{8\pi\varepsilon_0r^2}\dif r
    =\frac{q^2}{8\pi\varepsilon_0}\left(\frac{1}{b}-\frac{1}{a}\right)
\end{align*}
\subsection{导体}
\mysssec{导体内部电场为零}
若导体内部存在非零电场, 则自由电荷将在电场作用下持续运动, 
与静电平衡的假设矛盾。
因此, 导体内部必须满足
\[
\boldsymbol E = -\boldsymbol{\nabla}\varphi = \boldsymbol 0.
\]

\mysssec{导体内部体电荷密度为零}
\[
\boldsymbol{\nabla}\cdot\boldsymbol E = \frac{\rho}{\varepsilon_0},
\]
结合 $\boldsymbol E=\boldsymbol 0$,
\[
\rho = 0 \qquad (\text{导体内部}).
\]
\mysssec{净电荷只能分布在导体表面}

既然导体内部体电荷密度为零, 
而导体整体可能带有净电荷, 
则这些电荷只能分布在导体的表面上。
\mysssec{导体是等势体}
由 $\boldsymbol E = -\boldsymbol{\nabla}\varphi$ 以及
导体内部 $\boldsymbol E=\boldsymbol 0$, 
可知导体内部电势处处相同
\mysssec{导体表面外侧电场垂直于表面}

考虑导体表面上一点的切向电场分量。
若存在非零切向分量, 
则自由电荷将在表面沿切向运动, 
从而破坏静电平衡。

因此, 导体表面外侧的电场只能沿法向:
\[
\boldsymbol E_\parallel = 0
\]
\mysssec{空腔中含点电荷的球形导体}

设一不带电的球形导体, 半径为 $R$, 中心位于原点, 其内部挖去一任意形状的空腔。
在空腔内某处放置一点电荷 $q$。求球外区域的电场分布。

导体处于静电平衡时, 金属内部电场为零, 因而导体整体为等势体。
取一紧贴导体内壁、位于金属内部的高斯面, 由高斯定律可得
\[
\oint \boldsymbol E\cdot \dif \boldsymbol a = 0,
\]
故内壁所感应的总电荷必为 $-q$。
又由于导体整体不带电, 外表面所带电荷总量为 $+q$。

\medskip

关键在于确定外表面电荷的分布形式。
注意到:在所有满足
\[
\boldsymbol E = 0 \quad (\text{导体内部}), \qquad 
\varphi = \text{常数} \quad (\text{导体表面})
\]
的允许电荷分布中, 实际的静电平衡态对应于体系总静电能的极小值。

\medskip

球外区域不含自由电荷, 其电势完全由外表面电荷分布决定。
若外表面电荷分布破坏球对称性, 则球外电场中将出现非径向分量, 
从而在保持总电荷为 $q$ 的约束下增加电场能
\[
U=\frac{\varepsilon_0}{2}\int E^2\,\dif\tau.
\]
因此, 能量极小所对应的电荷分布必然保持球对称性。

\medskip

由此可知, 外表面电荷在球面上均匀分布, 
其产生的球外电场与位于球心的点电荷 $q$ 完全相同。
因此, 球外任意一点处的电场为
\[
\boldsymbol E(\boldsymbol r)
=
\frac{1}{4\pi\varepsilon_0}\frac{q}{r^2}\,{\boldsymbol e_r},
\qquad r>R.
\]

\medskip

可见, 空腔的形状以及点电荷在腔内的具体位置, 
均不影响球外区域的电场分布。

\mysssec{一个半径为R的金属球, 带有电荷$q$, 这个金属球又被一个厚的同心金属球壳所包围(球壳内径为$a$, 外径为$b$)。\\
(a)分别求出R, a, b球面上的电荷面密度$\sigma$。\\
(b)求出球心处的电势, 选无限远处为参考点。\\
(c)现在球壳的外表面接地, 电势能为零。(a)和(b)所得结果改变为什么?}

(a):
\begin{align*}
    \sigma_R&=\frac{q}{4\pi R^2}\\
    \sigma_a&=\frac{q}{4\pi a^2}\\
    \sigma_b&=\frac{q}{4\pi b^2}
\end{align*}

(b):
\begin{align*}
    U=\int_{b}^{\infty}\frac{q}{4\pi\varepsilon_0 r}\dif r
    +\int_{R}^{a}\frac{q}{4\pi\varepsilon_0 r}\dif r
    =\frac{q}{4\pi\varepsilon_0 }\ln \left(\frac{bR}{a}\right)
\end{align*}

(c):
\begin{align*}
    \sigma_R&=\frac{q}{4\pi R^2}\\
    \sigma_a&=\frac{q}{4\pi a^2}\\
    \sigma_b&=0\\
    U&=\frac{q}{4\pi\varepsilon_0 }\ln \left(\frac{R}{a}\right)
\end{align*}
\mysssec{一个半径为$R$的导体球体, 其内部有两个半径分别为a和b的圆形空洞, 在$a$空洞的中心放有点电荷$q$, 在$b$空洞的中心放有点电荷q。\\
(a)求出电荷面密度$\sigma_a$, $\sigma_b$和$\sigma_R$。\\
(b)导体外面的电场是什么?\\
(c)每个空洞内的电场是什么?\\
(d)$q_a$和$q_b$受到的力是什么?\\
(e)如果让第三个电荷$q$靠近导体, 上面所得结果哪一个会发生变化?}
(a):
\begin{align*}
    \sigma_R&=\frac{2q}{4\pi R^2}\\
    \sigma_a&=\frac{q}{4\pi a^2}\\
    \sigma_b&=\frac{q}{4\pi b^2}
\end{align*}

(b):
\begin{align*}
    \boldsymbol{E}=\frac{2q\boldsymbol{r}}{4\pi\varepsilon_0 r^3}
\end{align*}

(c):
\begin{align*}
    \boldsymbol{E_a}&=\frac{q\boldsymbol{r}}{4\pi\varepsilon_0 r^3}\\
    \boldsymbol{E_b}&=\frac{q\boldsymbol{r}}{4\pi\varepsilon_0 r^3}
\end{align*}

(d):0

(e): $\sigma_R, b$

\subsection{电容}
\begin{equation}
    C\equiv\frac{Q}{U}
\end{equation}
\begin{equation}
    W=\int_{0}^{Q}\left(\frac{q}{C}\right)\dif q=\frac{Q^2}{2C}
\end{equation}
\mysssec{两个同轴金属管壳, 半径分别为$a$和$b$, 求出单位长度的电容。}
\begin{align*}
    U&=\int_{a}^{b}\frac{Q}{2\varepsilon_0 r}\dif r\\
    &=\frac{Q}{2\varepsilon_0}\ln\left(\frac{b}{a}\right)\\
    C&=\frac{Q}{U}\\
    &=\frac{2\varepsilon_0}{\ln\left(\dfrac{b}{a}\right)}
\end{align*}
\subsection{拉普拉斯方程}
\begin{align}
    \frac{\dif^2 U}{\dif x_i^2}&=0
\end{align}
\mysssec{一维拉普拉斯方程}
\begin{align}
    U&=ax+b
\end{align}
\mysssec{在球坐标下,对$U$仅依赖于$r$的情况,求出拉普拉斯方程的一般解。对柱坐标系,假定$U$
仅依赖于$s$,做同样的计算。}
\begin{align*}
    \frac{1}{r^2}\frac{\partial}{\partial r}\left(r^2\dfrac{{\partial U}}{\partial r}\right)&=0\\
    \frac{\partial}{\partial r}\left(r^2\dfrac{{\partial U}}{\partial r}\right)&=0\\
        r^2\dfrac{{\partial U}}{\partial r}&=C_1\\
        \dfrac{{\partial U}}{\partial r}&=\frac{C_1}{r^2}\\
        U&=-\frac{C_1}{r}+C_2
\end{align*}
\begin{align*}
    \frac{1}{s}\frac{\partial}{\partial u_s}\left(s\dfrac{{\partial U}}{\partial u_s}\right)&=0\\
    s\dfrac{{\partial U}}{\partial u_s}&=C_1\\
    \dfrac{{\partial U}}{\partial u_s}&=\frac{C_1}{s}\\
    U&=C_1\ln s+C_2
\end{align*}
\mysssec{$U=\dfrac{1}{4\pi\varepsilon_0}\left[\dfrac{q}{\sqrt{r^2+a^2-2ra\cos\theta}}-\dfrac{q}{\sqrt{R^2+\left(\frac{ra}{R}\right)^2-2ra\cos\theta}}\right]$求出球面上的诱导电荷面密度。对其积分求出总诱
导电荷。计算这个构型的能量。}
\begin{align*}
    \sigma&=-\varepsilon_0\frac{\partial U}{\partial r}\\
    &=-\varepsilon_0\dfrac{\partial \dfrac{1}{4\pi\varepsilon_0}\left[\dfrac{q}{\sqrt{r^2+a^2-2ra\cos\theta}}-\dfrac{q}{\sqrt{R^2+\left(\frac{ra}{R}\right)^2-2ra\cos\theta}}\right]}{\partial r}\\
    &=-\dfrac{q}{4\pi}\left[-\dfrac{2r-2a\cos\theta}{2\sqrt{r^2+a^2-2ra\cos\theta}^3}+\dfrac{2\frac{ra}{R}-2a\cos\theta}{2\sqrt{R^2+\left(\frac{ra}{R}\right)^2-2ra\cos\theta}}\right]\\
    &\xlongequal{r=R}-\dfrac{q}{4\pi}\left[-\dfrac{2R-2a\cos\theta}{2\sqrt{R^2+a^2-2Ra\cos\theta}^3}+\dfrac{2\frac{Ra}{R}-2a\cos\theta}{2\sqrt{R^2+\left(\frac{Ra}{R}\right)^2-2Ra\cos\theta}}\right]\\
    &\xlongequal{r=R}-\dfrac{q}{4\pi}\left[\dfrac{a-R}{\sqrt{R^2+a^2-2Ra\cos\theta}^3}\right]
\end{align*}
\begin{align*}
    W&=\int_{\infty}^{a}q\boldsymbol{E}\cdot\dif \boldsymbol{l}\\
    &=\int_{\infty}^{a}\frac{q^2R}{4\pi\varepsilon_0r\left(r-\frac{R^2}{r}\right)^2}\dif r\\
    &=\int_{\infty}^{a}\frac{q^2R}{8\pi\varepsilon_0\left(r^2-{R^2}\right)^2}\dif r^2\\
    &=-\left.\frac{q^2R}{8\pi\varepsilon_0\left(r^2-{R^2}\right)}\right|_{\infty}^{a}\\
    &=-\frac{q^2R}{8\pi\varepsilon_0\left(a^2-{R^2}\right)}
\end{align*}
\mysssec{一条无限长均匀带电线,电荷线密度为$\lambda$,它距一个接地导体板距离为$d$。带电线
平行于$x$轴并位于$x$轴上方,导体板为$xy$平面\\
(a)求出导体板上方的电势。\\
(b)求出导体板上的诱导电荷的面密度。}
(a):
\begin{align*}
    U=\frac{\lambda}{2\pi\varepsilon_0|x_3-d|}\ln\left|x_3-d\right|
    -\frac{\lambda}{2\pi\varepsilon_0(x_3+d)}\ln\left(x_3+d\right)
\end{align*}

(b):
\begin{align*}
    E_3&=\frac{\lambda}{\pi\varepsilon_0d}\\
    \sigma&=\varepsilon_0E_3=\frac{\lambda}{\pi d}
\end{align*}
\mysssec{两个半无限大接地导体板一端相接形成一个直角。在它们之间的区域有一个点电荷$q$,计算这个区域内的电势。
作用在$q$上的力是什么?把$q$从无限远处移到所示位置需做多少功?假定两板形成的角度不是$\frac{\pi}{2}$,而是另
外的一些角度,你还能用镜像法求解问题吗?如果不能,对什么样的特殊角度仍然可以用镜像法求解?}
假设在$(a,a),(-a,-a)$处有电荷$q$, $(a,-a),(-a,a)$有电荷$-q$, 当$x=0,\, z=0$时
\begin{align*}
    U
    =&\frac{q}{4\pi}\left(\frac{1}{\sqrt{a^2+(y-a)^2}}
    +\frac{1}{\sqrt{a^2+(y+a)^2}}
    -\frac{1}{\sqrt{a^2+(y-a)^2}}
    -\frac{1}{\sqrt{a^2+(y+a)^2}}\right)=0
\end{align*}
\begin{align*}
    \boldsymbol{F}&=q\boldsymbol{E}=-\frac{q^2\boldsymbol e_1}{4\pi\varepsilon_0 a^2}
    -\frac{q^2\boldsymbol e_2}{4\pi\varepsilon_0 a^2}
    +\frac{\frac{\sqrt{2}}{2}q^2\boldsymbol e_1+\frac{\sqrt{2}}{2}q^2\boldsymbol e_2}{4\pi\varepsilon_0 2a^2}\\
    \left|\boldsymbol{F}\right|&=\left(\sqrt{2}-\frac{1}{2}\right)\frac{q^2}{4\pi\varepsilon_0 a^2}\\
    W&=\int_{\infty}^{a}\left(\sqrt{2}-\frac{1}{2}\right)\frac{\sqrt{2}q^2}{4\pi\varepsilon_0 r^2}\dif r\\
    &=-\left.\left(\sqrt{2}-\frac{1}{2}\right)\frac{\sqrt{2}q^2}{4\pi\varepsilon_0 r}\right|_{\infty}^{a}\\
    &=-\left(\sqrt{2}-\frac{1}{2}\right)\frac{\sqrt{2}q^2}{4\pi\varepsilon_0 a}
\end{align*}

在无源区域内,电势满足拉普拉斯方程:
\[
\nabla^2 V(x,y) = 0.
\]

在二维情况下,可引入复势
\[
W(z) = \Phi(x,y) + i \Psi(x,y),
\]
其中
\[
\Phi = V.
\]

$W(z)$ 为解析函数的充要条件是
$\Phi$ 与 $\Psi$ 满足 Cauchy--Riemann 条件,
而这等价于 $\Phi$ 满足拉普拉斯方程。
因此,求解二维静电问题等价于构造合适的解析函数 $W(z)$。


二维中,点电荷对应的 Green 函数为对数型:
\[
W_0(z) =
-\frac{q}{2\pi\epsilon_0} \ln (z - z_0),
\]
其电势为
\[
U(z) = -\frac{q}{2\pi\epsilon_0} \ln |z - z_0|.
\]

该表达式在 $z=z_0$ 处具有对数奇点,
对应于二维点电荷。


定义映射
\[
w = z^{\pi/\alpha}.
\]

该映射具有如下性质:
\begin{itemize}
  \item 若 $0 < \arg z < \alpha$,则 $0 < \arg w < \pi$;
  \item 楔形区域被映射为上半平面;
  \item $\arg z = 0, \alpha$ 被映射为实轴。
\end{itemize}

因此,楔形导体边界在 $w$ 平面中对应于接地的实轴。


在上半平面中,实轴接地,
位于 $w_0$($\Im w_0 > 0$)的点电荷的复势为
\[
W(w)
=
-\frac{q}{2\pi\epsilon_0}
\ln\frac{w - w_0}{w - \overline{w_0}}.
\]

该表达式满足:
\begin{itemize}
  \item 在上半平面内调和;
  \item 在实轴上 $|w-w_0| = |w-\overline{w_0}|$,
        因而 $V=0$;
  \item 在 $w=w_0$ 处具有正确的对数奇点。
\end{itemize}

这是由唯一性定理保证的解。



将 $w=z^{\pi/\alpha}$ 代回,得到楔形区域内的复势:
\[
W(z)
=
-\frac{q}{2\pi\epsilon_0}
\ln
\frac{
z^{\pi/\alpha} - z_0^{\pi/\alpha}
}{
z^{\pi/\alpha} - \overline{z_0}^{\,\pi/\alpha}
}.
\]

电势为其实部:
\[
U(z)
=
-\frac{q}{2\pi\epsilon_0}
\ln
\left|
\frac{
z^{\pi/\alpha} - z_0^{\pi/\alpha}
}{
z^{\pi/\alpha} - \overline{z_0}^{\,\pi/\alpha}
}
\right|.
\]

这是任意楔角 $\alpha$ 下的严格解。


若
\[
\alpha = \frac{\pi}{n}, \quad n \in \mathbb{N},
\]
则
\[
w = z^n
\]
是单值多项式映射。

利用因式分解:
\[
z^n - z_0^n
=
\prod_{k=0}^{n-1}
\left(z - z_0 e^{2\pi i k/n}\right),
\]
电势可写为有限和:
\[
U(z)
=
-\frac{q}{2\pi\epsilon_0}
\sum_{k=0}^{n-1}
\ln
\left|
\frac{
z - z_0 e^{2\pi i k/n}
}{
z - \overline{z_0} e^{2\pi i k/n}
}
\right|.
\]

这正对应于有限多个镜像电荷的叠加。

若 $\alpha \neq \pi/n$,则 $z^{\pi/\alpha}$ 为多值函数,
解析延拓将产生无限多个像点,
镜像法不再以有限求和形式成立。
\subsection{分离变量法}
\subsubsection{直角坐标系}
{两个无限大接地金属平板平行于$xz$平面放置,一个位于$y=0$,另一个位于$y=a$。在$x=0$
两板的左端点,被与两板绝缘的无限长带封闭,带子上维持特定的电势$U_0(y)$。求出这个“夹缝”中的电势。}

由于几何结构和边界条件在 $z$ 方向具有平移对称性,且 $U_0$ 与 $z$ 无关,
物理解必然满足
\[
\frac{\partial\Phi}{\partial z}=0 .
\]
因此问题严格退化为二维拉普拉斯方程:
\begin{align*}
    \frac{\partial^2 V}{\partial x^2}
    +\frac{\partial^2V}{\partial y^2}&=0
\end{align*}
其边界条件为
\begin{align*}
    U&\xlongequal{y=0}0\\
    U&\xlongequal{y=a}0\\
    U&\xlongequal{x=0}U_0(y)\\
    U&\xrightarrow{x\to\infty}0
\end{align*}

设解为
\[
U(x,y)=X(x)Y(y).
\]

代入 Laplace 方程得
\[
\frac{X''(x)}{X(x)} + \frac{Y''(y)}{Y(y)} = 0.
\]

因此两项必须分别等于常数,记为 $-k^2$:
\begin{align*}
Y''(y) + k^2 Y(y) &= 0, \\
X''(x) - k^2 X(x) &= 0. 
\end{align*}

非平凡解存在当且仅当

\[
k = \frac{n\pi}{a}, \quad n=1,2,3,\dots
\]

对应本征函数为
\[
Y_n(y)=\sin\frac{n\pi y}{a}.
\]

对每个 $k_n=n\pi/a$,横向方程为
\[
X_n''(x) - \left(\frac{n\pi}{a}\right)^2 X_n(x)=0.
\]

通解为
\[
X_n(x)=A_n e^{-(n\pi/a)x}+B_n e^{+(n\pi/a)x}.
\]

由远处边界条件,要求 $V\to0$,
\[
B_n=0.
\]


利用线性叠加原理,电势的一般解为
\[
V(x,y)
=
\sum_{n=1}^{\infty}
A_n
\sin\frac{n\pi y}{a}
\,e^{-(n\pi/a)x}.
\]

在 $x=0$ 处,要求
\[
V(0,y)=U_0(y).
\]

因此
\[
U_0(y)
=
\sum_{n=1}^{\infty}
A_n
\sin\frac{n\pi y}{a}.
\]

\[
{
A_n
=
\frac{2}{a}
\int_0^a
U_0(y)
\sin\frac{n\pi y}{a}
\,dy
}
\]




综上,夹缝区域中的电势为
\[
{
U
=
\sum_{n=1}^{\infty}
\left[
\frac{2}{a}
\int_0^a
U_0(y')
\sin\frac{n\pi y'}{a}
\dif y'
\right]
\sin\frac{n\pi y}{a}
e^{-(n\pi/a)x}
}
\]

若问题在 $z$ 方向不具平移对称性,可进一步设
\[
\Phi = X(x)Y(y)Z(z),
\]
并引入
\[
Z'' + \lambda^2 Z = 0,
\qquad
Z(z)=e^{i\lambda z},\quad \lambda\in\mathbb{R}.
\]

此时解的结构变为
\[
\Phi(x,y,z)
=
\sum_{n=1}^\infty
\int_{-\infty}^{\infty}
C_n(\lambda)
\sin\frac{n\pi y}{a}
\,e^{-\sqrt{(n\pi/a)^2+\lambda^2}\,x}
\,e^{i\lambda z}
\,d\lambda .
\]
\subsubsection{球坐标系}

在球坐标系中,拉普拉斯算符为
\begin{align*}
\boldsymbol\nabla^2 U
&=
\frac{1}{r^2}\frac{\partial}{\partial r}
\left(
r^2 \frac{\partial U}{\partial r}
\right)
+
\frac{1}{r^2\sin\theta}
\frac{\partial}{\partial\theta}
\left(
\sin\theta \frac{\partial U}{\partial\theta}
\right)
+
\frac{1}{r^2\sin^2\theta}
\frac{\partial^2 U}{\partial\varphi^2}
\end{align*}

设电势可以写成完全分离的形式,即$U=R(r)\Theta(\theta)\Phi(\varphi)$
\begin{align*}
    \frac{1}{r^2}\frac{\partial}{\partial r}
\left(
r^2 \frac{\partial U}{\partial r}
\right)
+
\frac{1}{r^2\sin\theta}
\frac{\partial}{\partial\theta}
\left(
\sin\theta \frac{U}{\partial\theta}
\right)
+
\frac{1}{r^2\sin^2\theta}
\frac{\partial^2 U}{\partial\varphi^2}&=0\\
\frac{\partial}{\partial r}
\left(
r^2 \frac{\partial U}{\partial r}
\right)
+
\frac{1}{\sin\theta}
\frac{\partial}{\partial\theta}
\left(
\sin\theta \frac{\partial U}{\partial\theta}
\right)
+
\frac{1}{\sin^2\theta}
\frac{\partial^2 U}{\partial\varphi^2}&=0\\
\frac{\partial}{\partial r}
\left(
r^2 \frac{\partial R\Theta\Phi}{\partial r}
\right)
+
\frac{1}{\sin\theta}
\frac{\partial}{\partial\theta}
\left(
\sin\theta \frac{\partial R\Theta\Phi}{\partial\theta}
\right)
+
\frac{1}{\sin^2\theta}
\frac{\partial^2 R\Theta\Phi}{\partial\varphi^2}&=0\\
\frac{1}{R r^2}\frac{\dif }{\dif r}
\left(
r^2 \frac{\dif R}{\dif r}
\right)
+
\frac{1}{\Theta r^2\sin\theta}
\frac{\dif }{\dif \theta}
\left(
\sin\theta \frac{\dif \Theta}{\dif \theta}
\right)
+
\frac{1}{\Phi r^2\sin^2\theta}
\frac{\dif ^2\Phi}{\dif \varphi^2}
&=0\\
-
\left[
\frac{1}{\Theta\sin\theta}
\frac{\dif }{\dif \theta}
\left(
\sin\theta \frac{\dif \Theta}{\dif \theta}
\right)
+
\frac{1}{\Phi\sin^2\theta}
\frac{\dif ^2\Phi}{\dif \varphi^2}
\right]
&=
\frac{1}{R}\frac{\dif }{\dif r}
\left(
r^2 \frac{\dif R}{\dif r}
\right)
\end{align*}



由于左边仅依赖 $r$,右边仅依赖角变量,
两边必须等于同一个常数。


引入分离常数 $l(l+1),\, l\in Z$

于是得到:
\begin{equation}
\frac{\dif }{\dif r}
\left(r^2\frac{\dif R}{\dif r}
\right)
-
{l(l+1)} R
=0.
\end{equation}
\begin{equation}
\frac{1}{\Theta\sin\theta}
\frac{\dif }{\dif \theta}
\left(
\sin\theta \frac{\dif \Theta}{\dif \theta}
\right)
+
\frac{1}{\sin^2\theta}
\frac{1}{\Phi}
\frac{\dif ^2\Phi}{\dif \varphi^2}
+
l(l+1)
=0.
\end{equation}

继续分离$\Theta(\theta)\Phi(\varphi)$,令
\[
\frac{1}{\Phi}\frac{\dif ^2\Phi}{\dif \varphi^2} = -m^2,
\qquad
m \in {Z}.
\]

于是得到:

{$\varphi$ 方程}
\begin{equation}
\frac{\dif ^2\Phi}{\dif \varphi^2} + m^2 \Phi = 0,
\end{equation}
其解为
\[
\Phi_m(\varphi) = C_1e^{i m \varphi}
\]

{$\theta$ 方程}
\begin{equation}
\frac{1}{\sin\theta}
\frac{\dif }{\dif \theta}
\left(
\sin\theta \frac{\dif \Theta}{\dif \theta}
\right)
+
\left[
l(l+1)
-
\frac{m^2}{\sin^2\theta}
\right]\Theta
=0.\label{theta方程}
\end{equation}

这是{关联勒让德方程},
其在 $\theta\in[0,\pi]$ 上正则的解为
\[
\Theta_{lm}(\theta)=P_{lm}(\cos\theta),\qquad
l=0,1,2,\dots,
\qquad
|m|\le l.
\]
其中,
\begin{equation}
    P_{lm}(x)\equiv\sqrt{(1+x^2)^m}\frac{\dif^m}{\dif x^m}P_l(x)
\end{equation}
$P_l(x)$由罗德里格(Rodrigue)公式定义:
\begin{equation}
    P_l(x)\equiv\frac{1}{2^ll!}\left(\frac{\dif}{\dif x}\right)^l(x^2-1)^l
\end{equation}
罗德里格公式显然仅对非负的整数$l$成立。另外,它仅提供给我们一个解。但是式\ref{theta方程}应当有两个解。情况是那些另外的解在$\theta=0$和/或$\theta=\pi$发散



为了方便,将角向部分合并,定义球谐函数
\begin{equation}
    Y_{lm}=\sqrt{\frac{2l+1}{4\pi}\frac{(l-m)!}{(l+m)!}}P_{lm}(\cos\theta) e^{i m \varphi}
\end{equation}

径向方程为
\[
r^2 R'' + 2r R' - l(l+1)R = 0,
\]

通解为
\begin{equation}
R_l(r)
=
A_l r^l
+
B_l r^{-(l+1)}
\end{equation}




将各部分组合,得到球坐标下拉普拉斯方程的一般解
\begin{equation}
U(r,\theta,\varphi)
=
\left(
A_{lm} r^l
+
B_{lm} r^{-(l+1)}
\right)
Y_{lm}(\theta,\varphi)
\end{equation}

当$m=0$时, 式\ref{theta方程}退化为
\begin{equation}
\frac{1}{\sin\theta}
\frac{\dif }{\dif \theta}
\left(
\sin\theta \frac{\dif \Theta}{\dif \theta}
\right)
+
l(l+1)\Theta
=0\label{退化theta方程}
\end{equation}

式\ref{theta方程}的解退化为
\begin{equation}
    \Theta(\theta)=P_l(\cos\theta)
\end{equation}

球坐标下拉普拉斯方程的一般解退化为

\begin{equation}
U(r,\theta)
=
\left(
A_l r^l
+
B_l r^{-(l+1)}
\right)
P_{l}(\cos\theta)
\end{equation}

\subsubsection{勒让德多项式性质的证明}
\paragraph{$l\in Z$的证明}
\begin{align*}
    0&=\frac{1}{\sin\theta}
\frac{\dif }{\dif \theta}
\left(
\sin\theta \frac{\dif \Theta}{\dif \theta}
\right)
+
l(l+1)\Theta\\
&\xlongequal{u=\cos\theta}\frac{1}{\sqrt{1-u^2}}
\frac{\dif }{\dif \arccos u}
\left(
\sqrt{1-u^2} \frac{\dif \Theta}{\dif \arccos u}
\right)
+
l(l+1)\Theta\\
&=
\frac{\dif }{\dif u}
\left[\left(1-u^2\right)\frac{\dif \Theta}{\dif u}
\right]
+
l(l+1)\Theta\\
&=
\frac{\dif }{\dif u}\left(\frac{\dif \Theta}{\dif u}\right)
-\frac{\dif }{\dif u}\left(u^2\frac{\dif \Theta}{\dif u}\right)
+
l(l+1)\Theta\\
&=
\frac{\dif^2 \Theta}{\dif u^2}
-u^2\frac{\dif^2 \Theta}{\dif u^2}
-2u\frac{\dif \Theta}{\dif u}
+l(l+1)\Theta
\end{align*}
设
\begin{align*}
    \Theta&=\sum\limits_{n=0}^{\infty}\frac{a_n}{n!}u^n\\
    0&=
\frac{\dif^2 \sum\limits_{n=0}^{\infty}\dfrac{a_n}{n!}u^n}{\dif u^2}
-u^2\frac{\dif^2 \sum\limits_{n=0}^{\infty}\dfrac{a_n}{n!}u^n}{\dif u^2}
-2u\frac{\dif \sum\limits_{n=0}^{\infty}\dfrac{a_n}{n!}u^n}{\dif u}
+l(l+1)\sum\limits_{n=0}^{\infty}\dfrac{a_n}{n!}u^n\\
&=
\sum\limits_{n=0}^{\infty}\dfrac{a_{n+2}}{n!}u^n
-u^2\sum\limits_{n=0}^{\infty}\dfrac{a_{n+2}}{n!}u^n
-2u\sum\limits_{n=0}^{\infty}\dfrac{a_{n+1}}{n!}u^n
+l(l+1)\sum\limits_{n=0}^{\infty}\dfrac{a_{n}}{n!}u^n\\
&=
\sum\limits_{n=0}^{\infty}\dfrac{a_{n+2}}{n!}u^n
-\sum\limits_{n=0}^{\infty}\dfrac{a_{n+2}}{n!}u^{n+2}
-2\sum\limits_{n=0}^{\infty}\dfrac{a_{n+1}}{n!}u^{n+1}
+l(l+1)\sum\limits_{n=0}^{\infty}\dfrac{a_{n}}{n!}u^n\\
&=
\sum\limits_{n=0}^{\infty}\dfrac{a_{n+2}}{n!}u^n
-\sum\limits_{n=2}^{\infty}\dfrac{a_{n}}{(n-2)!}u^{n}
-2\sum\limits_{n=1}^{\infty}\dfrac{a_{n}}{(n-1)!}u^{n}
+l(l+1)\sum\limits_{n=0}^{\infty}\dfrac{a_{n}}{n!}u^n\\
&=
\sum\limits_{n=0}^{\infty}\dfrac{a_{n+2}}{n!}u^n
-\sum\limits_{n=0}^{\infty}\dfrac{a_{n}}{(n-2)!}u^{n}
-2\sum\limits_{n=0}^{\infty}\dfrac{a_{n}}{(n-1)!}u^{n}
+l(l+1)\sum\limits_{n=0}^{\infty}\dfrac{a_{n}}{n!}u^n\, (n>1)\\
&=
\dfrac{a_{n+2}}{n!}
-\dfrac{a_{n}}{(n-2)!}
-2\dfrac{a_{n}}{(n-1)!}
+l(l+1)\dfrac{a_{n}}{n!}\\
&=
a_{n+2}
-a_{n}n(n-1)
-2a_{n}n
+l(l+1)a_{n}\\
a_{n+2}&=[n(n+1)-l(l+1)]a_n
\end{align*}
由高斯判别法可得级数在$u=\pm 1$时发散, 因此$l\in Z$
\paragraph{正交性的证明}
取两个不同阶数的勒让德函数 $P_a$、$P_b$,满足:
\begin{align*}
    0&=
\frac{\dif}{\dif u}\left[(1-u^2)P_a'\right]
+a(a+1)P_a\\
    0&=
\frac{\dif}{\dif u}\left[(1-u^2)P_b'\right]
+b(b+1)P_b\\
    0&=
\left\{\frac{\dif}{\dif u}\left[(1-u^2)P_a'\right]
+a(a+1)P_a\right\}P_b
-\left\{\frac{\dif}{\dif u}\left[(1-u^2)P_b'\right]
+b(b+1)P_b\right\}P_a
\\
&=\int_{-1}^{1}
\left\{\frac{\dif}{\dif u}\left[(1-u^2)P_a'\right]
+a(a+1)P_a\right\}P_b
-\left\{\frac{\dif}{\dif u}\left[(1-u^2)P_b'\right]
+b(b+1)P_b\right\}P_a\dif u
\\
&=\int_{-1}^{1}
\frac{\dif}{\dif u}\left[(1-u^2)P_a'\right]P_b
+a(a+1)P_aP_b
-\frac{\dif}{\dif u}\left[(1-u^2)P_b'\right]P_a
-b(b+1)P_bP_a\dif u
\\
&=\int_{-1}^{1}
\frac{\dif}{\dif u}\left[(1-u^2)P_a'\right]P_b
-\frac{\dif}{\dif u}\left[(1-u^2)P_b'\right]P_a\dif u
+\left[a(a+1)-b(b+1)\right]\int_{-1}^{1}P_aP_b
\dif u
\end{align*}
由分部积分可得
\begin{align*}
    \int_{-1}^{1}
\frac{\dif}{\dif u}\left[(1-u^2)P_a'\right]P_b
-\frac{\dif}{\dif u}\left[(1-u^2)P_b'\right]P_a\dif u
&=\left.(1-u^2)(P_bP_a'-P_aP_b')\right|_{-1}^{1}=0
\end{align*}
得证
\begin{align*}
    \left[a(a+1)-b(b+1)\right]\int_{-1}^{1}P_aP_b
\dif u=0
\dif u
\end{align*}
完备性的证明略过
\mysssec{两个无限长接地金属板,分别在$y=0$和$y=a$放置,在$x=±b$的侧边连接有电势为$U_0$的两个金属带。求出这个矩形管中的电势。}
此时边界条件为:
\begin{align*}
        U&\xlongequal{y=0}0\\
    U&\xlongequal{y=a}0\\
    U&\xlongequal{x=b}U_0\\
    U&\xlongequal{x=-b}U_0
\end{align*}
做法同前解得
\begin{align*}
    U=(Ae^{kx}+Be^{-kx})(C\sin \frac{n\pi}{a}x+D\cos \frac{n\pi}{a}x)
\end{align*}
因为$U(-x)=U(x),\, U\xlongequal{y=0}0$, 所以$A=B,\, D=0$, 并把系数吸进$C$得
\begin{align*}
    U=C\cosh\frac{n\pi}{a} x\sin \frac{n\pi}{a}y
\end{align*}
余下的事是构造一般的叠加解, 设定系数$C_n$,使其拟合边界条件
\begin{align*}
    U(b,y)&=\sum_{n=1}^{\infty}C_n\cosh\frac{n\pi}{a} b\sin \frac{n\pi}{a}y=U_0
\end{align*}
因为$U(b,y)=U(b,-y)$, 所以$\sin \frac{n\pi}{a}y$为偶函数, $n$为奇数, 即
\begin{align*}
    U(b,y)=\sum_{n=0}^{\infty}C_n\cosh\frac{2n+1}{a}\pi b\sin \frac{2n+1}{a}\pi y=U_0
\end{align*}
\begin{align*}
    C_n\cosh\frac{2n+1}{a}\pi b&=\frac{2}{a}\int_{0}^{a}U_0\sin \frac{2n+1}{a}\pi y\dif y\\
    &=\frac{4U_0}{(2n+1)\pi }\left(\cosh\frac{2n+1}{a}\pi b\right)^{-1}\\
    U&=\sum_{n=0}^{\infty}\frac{4U_0}{(2n+1)\pi }\left(\cosh\frac{2n+1}{a}\pi b\right)^{-1}\cosh\frac{n\pi}{a} x\sin \frac{n\pi}{a}y
\end{align*}
\mysssec{一个半径为$R$的球面上的电势为
$U_0= k\cos3\theta$。求出球面内外的电势以及球面上的电荷面密度$\sigma(\theta)$。(假定球内和球外没有电荷分布。)}
球内:
\begin{align*}
    U(R,\theta)&=k\cos3\theta\\
    \left(
A_{l} R^l
+
B_{l} R^{-(l+1)}
\right)
P_{l}(\cos\theta)
&=k\cos3\theta\\
A_{l} R^lP_{l}(\cos\theta)
&=k\cos3\theta\\
\int_{0}^{\pi}A_{l}^2 R^{2l}P_{l}^2(\cos\theta)\sin\theta\dif \theta
&=\int_{0}^{\pi}k\cos3\theta  A_{l}R^lP_{l}(\cos\theta)\sin\theta\dif \theta\\
\int_{0}^{\pi}A_{l} R^lP_{l}^2(\cos\theta)\sin\theta\dif \theta
&=\int_{0}^{\pi}k\cos3\theta  P_{l}(\cos\theta)\sin\theta\dif \theta\\
\int_{0}^{\pi}A_{0} R^{0}P_{0}^2(\cos\theta)\sin\theta\dif \theta
&=\int_{0}^{\pi}k\cos3\theta P_{0}(\cos\theta)\sin\theta\dif \theta\\
\int_{0}^{\pi}A_{0}\sin\theta\dif \theta
&=\int_{0}^{\pi}k\cos3\theta\sin\theta\dif \theta\\
A_{0}
&=0\\
\int_{0}^{\pi}A_{1} RP_{1}^2(\cos\theta)\sin\theta\dif \theta
&=\int_{0}^{\pi}k\cos3\theta P_{1}(\cos\theta)\sin\theta\dif \theta\\
\int_{0}^{\pi}A_{1} R\cos^2\theta\sin\theta\dif \theta
&=\int_{0}^{\pi}k\cos3\theta \cos\theta\sin\theta\dif \theta
\end{align*}
不想算积分了
\mysssec{假定一个球面上的电势为$U_0(\theta)$,并且球内球外没有电荷分布。证明球面上的电荷面密度为$\sigma=\frac{\varepsilon_0}{2R}(2l+1)^2P_l(\cos\theta)\int_{0}^{\pi}U_0(\theta')P_l(\cos\theta')\sin\theta'\dif\theta'$}
\begin{align*}
    U(R,\theta)&=U_0(\theta)\\
    \left(
A_{l} R^l
+
B_{l} R^{-(l+1)}
\right)
P_{l}(\cos\theta)
&=U_0(\theta)
\end{align*}
球外:
\begin{align*}
B_{l} R^{-(l+1)}
P_{l}(\cos\theta)
&=U_0(\theta)\\
\int_{0}^{\pi}B_{l}^2 R^{-2(l+1)}P_{l}^2(\cos\theta)\sin\theta\dif\theta
&=\int_{0}^{\pi}B_{l} R^{-(l+1)}P_{l}(\cos\theta)U_0(\theta)\sin\theta\dif\theta\\
\int_{0}^{\pi}B_{l} R^{-(l+1)}P_{l}^2(\cos\theta)\sin\theta\dif\theta
&=\int_{0}^{\pi}P_{l}(\cos\theta)U_0(\theta)\sin\theta\dif\theta\\
B_{l}
&=\dfrac{\int_{0}^{\pi}P_{l}(\cos\theta)U_0(\theta)\sin\theta\dif\theta}{\int_{0}^{\pi} R^{-(l+1)}P_{l}^2(\cos\theta)\sin\theta\dif\theta}
\end{align*}
球内:
\begin{align*}
    A_{l} R^lP_{l}(\cos\theta)
&=U_0(\theta)\\
    \int_{0}^{\pi}A_{l} R^lP_{l}^2(\cos\theta)\sin\theta\dif\theta
&=\int_{0}^{\pi}U_0(\theta)P_{l}(\cos\theta)\sin\theta\dif\theta\\
A_l&=\frac{\int_{0}^{\pi}U_0(\theta)P_{l}(\cos\theta)\sin\theta\dif\theta}{\int_{0}^{\pi} R^lP_{l}^2(\cos\theta)\sin\theta\dif\theta}
\end{align*}
由连续性可得
\begin{align*}
    A_{l} R^lP_{l}(\cos\theta)&=B_{l} R^{-(l+1)}
P_{l}(\cos\theta)\\
    A_{l} R^l&=B_{l} R^{-(l+1)}
\end{align*}
\begin{align*}
    \sigma&=-\varepsilon_0\frac{\partial U}{\partial r}\\
    \frac{\varepsilon_0}{2R}(2l+1)^2P_l(\cos\theta)\int_{0}^{\pi}U_0(\theta')P_l(\cos\theta')\sin\theta'\dif\theta'&=-\varepsilon_0\frac{\partial U}{\partial r}\\
        \frac{1}{2R}(2l+1)^2P_l(\cos\theta)\int_{0}^{\pi}U_0(\theta')P_l(\cos\theta')\sin\theta'\dif\theta'
        &=-\left.\frac{\partial B_{l} r^{-(l+1)}
P_{l}(\cos\theta)}{\partial r}+\frac{\partial A_{l} r^{l}
P_{l}(\cos\theta)}{\partial r}\right|_{r=R}\\
        \frac{1}{2R}(2l+1)^2P_l(\cos\theta)\int_{0}^{\pi}U_0(\theta')P_l(\cos\theta')\sin\theta'\dif\theta'&=(l+1)B_{l} R^{-(l+2)}
P_{l}(\cos\theta)
+lA_lR^{l-1}P_{l}(\cos\theta)\\
        \frac{1}{2}(2l+1)^2P_l(\cos\theta)\int_{0}^{\pi}U_0(\theta')P_l(\cos\theta')\sin\theta'\dif\theta'&=(l+1)B_{l} R^{-(l+1)}
P_{l}(\cos\theta)
+lA_lR^lP_{l}(\cos\theta)\\
        \frac{1}{2}(2l+1)^2P_l(\cos\theta)\int_{0}^{\pi}U_0(\theta')P_l(\cos\theta')\sin\theta'\dif\theta'&=(l+1)B_{l} R^{-(l+1)}
P_{l}(\cos\theta)
+lB_{l} R^{-(l+1)}P_{l}(\cos\theta)\\
        \frac{1}{2}(2l+1)^2P_l(\cos\theta)\int_{0}^{\pi}U_0(\theta')P_l(\cos\theta')\sin\theta'\dif\theta'&=(2l+1)B_{l} R^{-(l+1)}
P_{l}(\cos\theta)
\end{align*}
逐项比较可得(不使用爱因斯坦求和约定)
\begin{align*}
\frac{1}{2}(2l+1)^2P_l(\cos\theta)\int_{0}^{\pi}U_0(\theta')P_l(\cos\theta')\sin\theta'\dif\theta'&=(2l+1)B_{l} R^{-(l+1)}
P_{l}(\cos\theta)\\
\frac{1}{2}(2l+1)\int_{0}^{\pi}U_0(\theta')P_l(\cos\theta')\sin\theta'\dif\theta'&=B_{l} R^{-(l+1)}\\
\frac{1}{2}(2l+1)\int_{0}^{\pi}U_0(\theta')P_l(\cos\theta')\sin\theta'\dif\theta'&=\dfrac{\int_{0}^{\pi}P_{l}(\cos\theta)U_0(\theta)\sin\theta\dif\theta}{\int_{0}^{\pi} R^{-(l+1)}P_{l}^2(\cos\theta)\sin\theta\dif\theta} R^{-(l+1)}\\
\frac{1}{2}(2l+1)&=\dfrac{1}{\int_{0}^{\pi} P_{l}^2(\cos\theta)\sin\theta\dif\theta} 
\end{align*}
\mysssec{一个带电金属球(电荷为$Q$,半径为$R$)置于均匀外电场$\boldsymbol E_0$中, 求出球外的电势。}
边界条件:
\begin{align*}
    \lim_{r\to\infty}U&\to E_0r\cos\theta\\
    U(R)&=0\\
    \left(
A_{l} R^l
+
B_{l} R^{-(l+1)}
\right)
P_{l}(\cos\theta)&=0\\
    \left(
A_{l} R^l
+
B_{l} R^{-(l+1)}
\right)
P_{l}(\cos\theta)&=0
\end{align*}
显然$A_1 =-E_0$。,其余诸$A_l$为零
\begin{align*}
-E_0R^1
+
B_{1} R^{-2}
&=0\\
B_1&=E_0R^3
\end{align*}
\subsection{习题}
\mysssec{一个均匀带电、边长为$2a$的正方形面, 电荷面密度为$\sigma$。
求出距中心高度为$z$处的电场。}
\begin{align*}
    &\int_{0}^{a}\frac{\sigma}{\pi\varepsilon_0\sqrt{2L^2+z^2}}\boldsymbol e_3\frac{2Lz}{{L^2+z^2}}\dif L\\
    \xlongequal{2L^2=z^2\tan^2\theta}&\int_{0}^{\arctan\frac{a}{\sqrt{2}z}}\frac{\sigma}{\pi\varepsilon_0\sqrt{z^2\tan^2\theta+z^2}}\boldsymbol e_3\frac{2z\tan\theta z}{{z^2\tan^2\theta+2z^2}}\dif z\tan\theta\\
    =&\int_{0}^{\arctan\frac{\sqrt{2}a}{z}}\frac{\sigma\cos\theta}{\pi\varepsilon_0z\cos^2\theta}\boldsymbol e_3\frac{2z\tan\theta z}{{z^2\tan^2\theta+2z^2}}\dif z\theta\\
    =&\int_{0}^{\arctan\frac{\sqrt{2}a}{z}}\frac{\sigma}{\pi\varepsilon_0\cos\theta}\boldsymbol e_3\frac{2\tan\theta}{{\tan^2\theta+2}}\dif \theta\\
    =&\int_{0}^{\arctan\frac{\sqrt{2}a}{z}}\frac{2\sigma}{\pi\varepsilon_0 }\boldsymbol e_3\frac{\tan\theta}{{\tan^2\theta\cos\theta+2\cos\theta}}\dif \theta\\
    =&\int_{0}^{\arctan\frac{\sqrt{2}a}{z}}\frac{2\sigma}{\pi\varepsilon_0 }\boldsymbol e_3\frac{\sin\theta}{{\sin^2\theta+2\cos^2\theta}}\dif \theta\\
    =&-\int_{0}^{\arctan\frac{\sqrt{2}a}{z}}\frac{2\sigma}{\pi\varepsilon_0 }\boldsymbol e_3\frac{1}{{1+\cos^2\theta}}\dif \cos\theta\\
    =&-\left.\frac{2\sigma}{\pi\varepsilon_0 }\boldsymbol e_3\arctan\cos\theta\right|_{0}^{\arctan\frac{\sqrt{2}a}{z}}\\
    =&\frac{2\sigma}{\pi\varepsilon_0 }\boldsymbol e_3\left(\frac{\pi}{4}-\arctan\frac{z}{\sqrt{2a^2+z^2}}\right)
\end{align*}
\mysssec{已知电场$\boldsymbol{E}=\dfrac{A\boldsymbol{e}_r+B\sin\theta\cos\phi\boldsymbol{e}_\phi}{r}$, 求电荷密度}
\begin{align*}
    \boldsymbol{\nabla}\cdot\boldsymbol{E}=
    &=\frac{1}{sr}\dfrac{\partial srE_r}{\partial r}
    +\frac{1}{s}\dfrac{\partial E_{\phi}}{\partial \phi}\\
    &=\frac{1}{sr}\dfrac{\partial sA}{\partial r}
    +\frac{1}{sr}\dfrac{\partial B\sin\theta\cos\phi}{\partial \phi}\\
    &=\frac{A}{r^2}-\frac{1}{r^2}{B\sin\phi}
\end{align*}
\mysssec{一个均匀带电球体, 求出南半球与北半球之间的净相互作用力}
\begin{align*}
    E_z&=E_r\cos\theta
    =\frac{\rho\cos\theta}{4\pi\varepsilon_0r^2}\frac{4\pi r^3}{3}
    =\frac{\rho\cos\theta r}{3\varepsilon_0}\\
    F&=\int_{0}^{R}\dif r\int_{0}^{2\pi}\dif \phi\int_{0}^{\frac{\pi}{2}}r^2\sin\theta \rho E_z\dif\theta\\
    &=\int_{0}^{R}\dif r\int_{0}^{2\pi}\dif \phi\int_{0}^{\frac{\pi}{2}}r^3\sin\theta \rho\frac{\rho\cos\theta}{3\varepsilon_0}\dif\theta\\
    &=\pi\int_{0}^{R}r^3\dif r\int_{0}^{\frac{\pi}{2}}\sin\theta \rho^2\frac{2\cos\theta}{3\varepsilon_0}\dif\theta\\
    &=\pi R^4\int_{0}^{\frac{\pi}{2}}\sin\theta \rho^2\frac{\cos\theta}{6\varepsilon_0}\dif\theta\\
    &=\pi R^4\int_{0}^{\frac{\pi}{2}}\rho^2\frac{\sin2\theta}{24\varepsilon_0}\dif2\theta\\
    &=-\left.\rho^2R^4\pi\frac{\cos2\theta}{24\varepsilon_0}\right|_{0}^{\frac{\pi}{2}}\\
    &=\frac{\rho^2R^4\pi}{12\varepsilon_0}
\end{align*}
\mysssec{一个半径为$R$的倒置半球面均匀带电, 电荷面密度为$\sigma$。求出北极与球心处的电势差}
\begin{align*}
    U_{\text{半球北极}}&=\dif\varphi\int_{0}^{\frac{\pi}{2}}\frac{\sigma R\sin\theta}{2\varepsilon_0}\frac{1}{\sqrt{R^2\left(1-\cos\theta\right)^2+R^2\sin^2\theta}}R\dif\theta\\
    &=\int_{0}^{\frac{\pi}{2}}\frac{\sigma \sin^2\theta}{2\varepsilon_0}\frac{1}{\sqrt{\left(1-\cos\theta\right)^2+\sin^2\theta}}R^2\dif\theta\\
    &=\int_{0}^{\frac{\pi}{2}}\frac{\sigma \sin\theta}{2\varepsilon_0}\frac{1}{\sqrt{2-2\cos\theta}}R\dif\theta\\
    &=\int_{0}^{\frac{\pi}{2}}\dfrac{\sigma 2\sin\dfrac{\theta}{2}\cos\dfrac{\theta}{2}}{2\sqrt{2}\varepsilon_0}\dfrac{1}{\sqrt{1-\cos^2\dfrac{\theta}{2}+\sin^2\dfrac{\theta}{2}}}R\dif\theta\\
    &=\int_{0}^{\frac{\pi}{2}}\dfrac{\sigma \sin\dfrac{\theta}{2}\cos\dfrac{\theta}{2}}{\sqrt{2}\varepsilon_0}\dfrac{1}{\sqrt{2\sin^2\dfrac{\theta}{2}}}R\dif\theta\\
    &=\int_{0}^{\frac{\pi}{2}}\dfrac{\sigma \cos\dfrac{\theta}{2}}{2\varepsilon_0}R\dif\theta\\
    &=\int_{0}^{\frac{\pi}{2}}\dfrac{\sigma}{\varepsilon_0}R\dif\sin\dfrac{\theta}{2}\\
    &=\left.\dfrac{\sigma \sin\dfrac{\theta}{2}}{\varepsilon_0}R\right|_{0}^{\frac{\pi}{2}}\\
    &=\dfrac{\sqrt{2}\sigma }{2\varepsilon_0}R\\
    U_{\text{半球球心}}&=\frac{2\pi R^2\sigma}{4\pi\varepsilon_0 R}
    =\frac{\sigma R}{2\varepsilon_0 }\\
    \Delta U&=\frac{\sigma }{2\varepsilon_0}\frac{R^2}{\sqrt{R^2}}
    +\frac{\sigma R}{2\varepsilon_0 }
    -\frac{\sigma }{2\varepsilon_0}\frac{R^2}{R+\sqrt{R^2+R^2}}
    -\dfrac{\sqrt{2}\sigma }{2\varepsilon_0}R\\
    &=\frac{\sigma R}{2\varepsilon_0 }
    -\dfrac{\sqrt{2}\sigma }{2\varepsilon_0}R\\
    &=\frac{\sigma R}{2\varepsilon_0 }
    -\dfrac{\sqrt{2}\sigma }{2\varepsilon_0}R
\end{align*}
\mysssec{一个半径为$R$的球体, 电荷密度$\rho=kr$, 求能量}
\begin{align*}
    \left|\boldsymbol{E}\right|&=\left\{\begin{aligned}
        &\frac{1}{4\pi\varepsilon_0 x_3^2}\int_{0}^{R}4\pi kr^3\dif r\,(x_3\ge R)\\
        &\frac{1}{4\pi\varepsilon_0 x_3^2}\int_{0}^{x_3}4\pi kr^3\dif r\,(x_3<R)
    \end{aligned}\right.\\
    &=\left\{\begin{aligned}
        &\frac{1}{4\varepsilon_0 x_3^2}kR^4\,(x_3\ge R)\\
        &\frac{1}{4\varepsilon_0 x_3^2}kx_3^4\,(x_3<R)
    \end{aligned}\right.\\
    &=\left\{\begin{aligned}
        &\frac{kR^4}{4\varepsilon_0 x_3^2}\,(x_3\ge R)\\
        &\frac{kx_3^2}{4\varepsilon_0}\,(x_3<R)
    \end{aligned}\right.\\
    U&=\frac{\varepsilon_0}{2}4\pi\int_{0}^{R}\left(\frac{kx_3^2}{4\varepsilon_0}\right)^2x_3^2\dif x_3
    +\frac{\varepsilon_0}{2}4\pi\int_{R}^{\infty}\left(\frac{kR^4}{4\varepsilon_0 x_3^2}\right)^2x_3^2\dif x_3\\
    &=2\pi\int_{0}^{R}\frac{k^2x_3^6}{16\varepsilon_0}\dif x_3
    +2\pi\int_{R}^{\infty}\frac{k^2R^8}{16\varepsilon_0 x_3^2}\dif x_3\\
    &=\pi\frac{k^2R^7}{56\varepsilon_0}
    +\pi\frac{k^2R^8}{8\varepsilon_0 R}\\
    &=\pi\frac{k^2R^7}{7\varepsilon_0}
\end{align*}
\mysssec{电势为$U=\frac{Ae^{-\lambda r}}{r}$, 求电场, 电荷密度, 总电荷}
\begin{align*}
    \boldsymbol{E}&=-\boldsymbol{\nabla}U\\
    &=\dfrac{\partial \dfrac{Ae^{-\lambda r}}{r}}{\partial r}\boldsymbol{e}_r\\
    &=\dfrac{\lambda rAe^{-\lambda r}+Ae^{-\lambda r}}{r^2}\boldsymbol{e}_r\\
    \rho&=\varepsilon_0\boldsymbol{\nabla}^2U\\
    &=\varepsilon_0\frac{1}{r^2}\dfrac{\partial r^2a_r}{\partial r}\\
    &=\varepsilon_0\frac{1}{r^2}\dfrac{\partial \lambda rAe^{-\lambda r}+Ae^{-\lambda r}}{\partial r}\\
    &=\varepsilon_0\dfrac{\lambda Ae^{-\lambda r}
    -\lambda^2 rAe^{-\lambda r}
    -\lambda Ae^{-\lambda r}}{r^2}\\
    &=-\dfrac{\varepsilon_0
    \lambda^2 Ae^{-\lambda r}
    }{r}\, (r\neq 0)\\
    \rho&=\dfrac{\varepsilon_0
    A4\pi\delta^{(3)}(r)
    }{r}
    -\dfrac{\varepsilon_0
    \lambda^2 Ae^{-\lambda r}
    }{r}\\
    Q&=4\pi\int_{0}^{\infty}\rho r^2\dif r\\
    &=0
\end{align*}
\mysssec{两条平行于z轴的无限长均匀带电线, 电荷线密度分别为$+\lambda$和$-\lambda$, 距离为$2d$。\\
(a)求出任意一点的电势。\\
(b)证明等势面为圆柱面, 对给定的电势$U$, 给出圆柱面的半径和轴的位置。}
% \begin{align*}
%     \boldsymbol E_{+\lambda}&=\frac{\lambda(\boldsymbol{r}-d\boldsymbol{e}_1)}{2\varepsilon_0(\boldsymbol{r}-d\boldsymbol{e}_1)\cdot(\boldsymbol{r}-d\boldsymbol{e}_1)}\\
%     &=\frac{\lambda(\boldsymbol{r}-d\boldsymbol{e}_1)}{2\varepsilon_0({r^2}-2d\boldsymbol{r}\cdot\boldsymbol{e}_1+d^2)}\\
%     \boldsymbol E_{-\lambda}&=\frac{\lambda(\boldsymbol{r}+d\boldsymbol{e}_1)}{2\varepsilon_0({r^2}+2d\boldsymbol{r}\cdot\boldsymbol{e}_1+d^2)}\\
%     \boldsymbol E&=\frac{\lambda(\boldsymbol{r}-d\boldsymbol{e}_1)}{2\varepsilon_0({r^2}-2d\boldsymbol{r}\cdot\boldsymbol{e}_1+d^2)}+\frac{\lambda(\boldsymbol{r}+d\boldsymbol{e}_1)}{2\varepsilon_0({r^2}+2d\boldsymbol{r}\cdot\boldsymbol{e}_1+d^2)}\\
%     &=\frac{\lambda(\boldsymbol{r}-d\boldsymbol{e}_1)({r^2}+2d\boldsymbol{r}\cdot\boldsymbol{e}_1+d^2)
%     +\lambda(\boldsymbol{r}+d\boldsymbol{e}_1)({r^2}-2d\boldsymbol{r}\cdot\boldsymbol{e}_1+d^2)}{2\varepsilon_0({r^2}-2d\boldsymbol{r}\cdot\boldsymbol{e}_1+d^2)({r^2}+2d\boldsymbol{r}\cdot\boldsymbol{e}_1+d^2)}\\
%     &=\lambda\frac{(\boldsymbol{r}-d\boldsymbol{e}_1)({r^2}+2d\boldsymbol{r}\cdot\boldsymbol{e}_1+d^2)
%     +(\boldsymbol{r}+d\boldsymbol{e}_1)({r^2}-2d\boldsymbol{r}\cdot\boldsymbol{e}_1+d^2)}
%     {2\varepsilon_0[({r^2}+d^2)^2-(2d\boldsymbol{r}\cdot\boldsymbol{e}_1)^2]}\\
%     &\xlongequal{u=({r^2}+d^2)}\lambda\frac{(\boldsymbol{r}-d\boldsymbol{e}_1)(u+2d\boldsymbol{r}\cdot\boldsymbol{e}_1)
%     +(\boldsymbol{r}+d\boldsymbol{e}_1)(u-2d\boldsymbol{r}\cdot\boldsymbol{e}_1)}
%     {2\varepsilon_0[u^2-(2d\boldsymbol{r}\cdot\boldsymbol{e})^2]}\\
%     &=\lambda\frac{\boldsymbol{r}(u+2d\boldsymbol{r}\cdot\boldsymbol{e}_1)
%     -d\boldsymbol{e}_1(u+2d\boldsymbol{r}\cdot\boldsymbol{e}_1)
%     +\boldsymbol{r}(u-2d\boldsymbol{r}\cdot\boldsymbol{e}_1)
%     +d\boldsymbol{e}_1(u-2d\boldsymbol{r}\cdot\boldsymbol{e}_1)}
%     {2\varepsilon_0[u^2-(2d\boldsymbol{r}\cdot\boldsymbol{e}_1)^2]}\\
%     &=\lambda\frac{2\boldsymbol{r}u
%     -2d\boldsymbol{e}_12d\boldsymbol{r}\cdot\boldsymbol{e}_1}
%     {2\varepsilon_0[u^2-(2d\boldsymbol{r}\cdot\boldsymbol{e}_1)^2]}\\
%     &=\lambda\frac{\boldsymbol{r}u
%     -2d\boldsymbol{e}_1d\boldsymbol{r}\cdot\boldsymbol{e}_1}
%     {\varepsilon_0[u^2-(2d\boldsymbol{r}\cdot\boldsymbol{e}_1)^2]}\\
% \end{align*}
(a):
\begin{align*}
    U_{+\lambda}&=-\frac{\lambda}{2\pi\varepsilon_0}\ln\left|\boldsymbol{r}-d\boldsymbol{e}_1\right|\\
    U_{-\lambda}&=\frac{\lambda}{2\pi\varepsilon_0}\ln\left|\boldsymbol{r}+d\boldsymbol{e}_1\right|\\
    U&=-\frac{\lambda}{2\pi\varepsilon_0}\ln\left|\boldsymbol{r}-d\boldsymbol{e}_1\right|
    +\frac{\lambda}{2\pi\varepsilon_0}\ln\left|\boldsymbol{r}+d\boldsymbol{e}_1\right|\\
    &=\frac{\lambda}{4\pi\varepsilon_0}\ln\left[\frac{(x_1+d)^2+x_2^2}{(x_1-d)^2+x_2^2}\right]
\end{align*}
(b):
\begin{align*}
    (x_1+d)^2+x_2^2&=k\left[(x_1-d)^2+x_2^2\right]\\
    x_1^2+2x_1d+d^2+x_2^2&=kx_1^2-2kx_1d+kd^2+kx_2^2\\
    0&=(k-1)x_1^2-2(k+1)x_1d+(k-1)d^2+(k-1)x_2^2\\
    0&\xlongequal{u=\frac{k+1}{k-1}}x_1^2-2ux_1d+d^2+x_2^2\\
    u^2-d^2&=(x_1-u)^2+x_2^2\\
    R&=\sqrt{u^2-d^2}\\
    &=\sqrt{\left(\frac{k+1}{k-1}\right)^2-d^2}\\
    &=\sqrt{\left(\frac{e^\dfrac{4\pi\varepsilon_0U}{\lambda}+1}{e^\dfrac{4\pi\varepsilon_0U}{\lambda}-1}\right)^2-d^2}
\end{align*}
\mysssec{在一个真空二极管中, 电子从阴极面“热蒸发”后向阳极面加速运动, 阴极电势为零, 对面阳极的电势为$U_0$。在两极间隙中所形成的电子云(称为空间电荷)很快会达到一种分布状态, 使得阴极面上的电场为零。然后在两极板之间形成稳定的电流I。假定两个极板面积$A$远大于它们之间的距离$d$($A>>d$), 所以边界效应可以忽略。则$U, \rho, v$(电子速度)都仅是$x$的函数。\\
(a)写出在两极板之间空间的泊松方程。\\
(b)假定电子从阴极是从静止开始运动的, 那么在点$x$, 这里电势为$U(x)$, 电子速度为多少?\\
(c)在稳定状态下, 电流$I$不依赖于$x$。那么$\rho$和$v$之间的关系是什么?\\
(d)利用上面的结果, 消去$\rho$和$v$, 得出$U$满足的微分方程。\\
(e)作为$x, U_0, d$的函数, 求出$U$的解。并与没有空间电荷的情况比较。另外作为$x$的函
数, 求出$\rho$和$v$。\\
(f) 证明$1 =KU^\frac{3}{2}_0$
求出常数$K$。}
(a):
\begin{align*}
    \frac{\partial^2 U}{\partial x^2}=\frac{\rho}{\varepsilon_0}
\end{align*}

(b):
\begin{align*}
    eU&=\frac{1}{2}mv^2\\
    v&=\sqrt{\frac{2eU}{m}}
\end{align*}

(c):
\begin{align*}
    I&=\frac{\Delta q}{\Delta t}\\
    &=\rho A v
\end{align*}

(d):
\begin{align*}
    \frac{\partial^2 U}{\partial x^2}&=\frac{\rho}{\varepsilon_0}\\
    \frac{\partial^2 U}{\partial x^2}&=\frac{I}{Av\varepsilon_0}\\
    \frac{\partial^2 U}{\partial x^2}&=\frac{I}{A\sqrt{\frac{2eU}{m}}\varepsilon_0}
    \end{align*}

    (e):
    \begin{align*}
    2\frac{\partial U}{\partial x}\frac{\partial^2 U}{\partial x^2}&=2\frac{\partial U}{\partial x}\frac{I}{A\sqrt{\frac{2eU}{m}}\varepsilon_0}\\
    \frac{\partial }{\partial x}\left(\frac{\partial U}{\partial x}\right)^2&=4\frac{\partial U^\frac{1}{2}}{\partial x}\frac{I}{A\sqrt{\frac{2e}{m}}\varepsilon_0}\\
    \left(\frac{\partial U}{\partial x}\right)^2&=4\frac{IU^\frac{1}{2}}{A\sqrt{\frac{2e}{m}}\varepsilon_0}\\
    \frac{\partial U}{\partial x}&=2\frac{I^\frac{1}{2}U^\frac{1}{4}}{A^\frac{1}{2}\left(\frac{2e}{m}\right)^\frac{1}{4}\varepsilon_0^\frac{1}{2}}\\
    U^{-\frac{1}{4}}{\partial U}{}&=2\frac{I^\frac{1}{2}{m}^\frac{1}{4}\partial x}{A^\frac{1}{2}(2e)^\frac{1}{4}\varepsilon_0^\frac{1}{2}}\\
    \frac{4}{3}U^{\frac{3}{4}}&=2\frac{I^\frac{1}{2}{m}^\frac{1}{4}x}{A^\frac{1}{2}(2e)^\frac{1}{4}\varepsilon_0^\frac{1}{2}}\\
    U&=\frac{3^{\frac{4}{3}}I^\frac{2}{3}x^{\frac{4}{3}}{m}^\frac{1}{4}}{2A^\frac{1}{2}e^\frac{1}{3}\varepsilon_0^\frac{2}{3}}\\
    v&=\sqrt{\frac{2eU}{m}}\\
    &=\sqrt{\dfrac{2e\dfrac{3^{\frac{4}{3}}I^\frac{2}{3}x^{\frac{4}{3}}{m}^\frac{1}{4}}{2A^\frac{1}{2}e^\frac{1}{3}\varepsilon_0^\frac{2}{3}}}{m}}\\
    &=\sqrt{\dfrac{3^{\frac{4}{3}}I^\frac{2}{3}x^{\frac{4}{3}}e^\frac{2}{3}}{A^\frac{1}{2}\varepsilon_0^\frac{2}{3}{m}^\frac{3}{4}}}\\
    &=\dfrac{3^{\frac{2}{3}}I^\frac{1}{3}x^{\frac{2}{3}}e^\frac{1}{3}}{A^\frac{1}{4}\varepsilon_0^\frac{1}{3}{m}^\frac{3}{8}}\\
    \rho&=\varepsilon_0\frac{\partial^2 U}{\partial x^2}\\
    &=\frac{2I^\frac{2}{3}{m}^\frac{1}{4}\varepsilon_0^\frac{1}{3}}{3^{\frac{2}{3}}A^\frac{1}{2}e^\frac{1}{3}x^{\frac{2}{3}}}
\end{align*}

(f):
\begin{align*}
    U&=\frac{3^{\frac{4}{3}}I^\frac{2}{3}x^{\frac{4}{3}}{m}^\frac{1}{4}}{2A^\frac{1}{2}e^\frac{1}{3}\varepsilon_0^\frac{2}{3}}\\
    I^\frac{2}{3}&=\frac{U2A^\frac{1}{2}e^\frac{1}{3}\varepsilon_0^\frac{2}{3}}{3^{\frac{4}{3}}x^{\frac{4}{3}}{m}^\frac{1}{4}}\\
    I&=\frac{U^\frac{3}{2}2^\frac{3}{2}A^\frac{3}{4}e^\frac{1}{2}\varepsilon_0^\frac{2}{3}}{3^2x^2{m}^\frac{3}{2}}\\
    I&=\frac{U_0^\frac{3}{2}2^\frac{3}{2}A^\frac{3}{4}e^\frac{1}{2}\varepsilon_0^\frac{2}{3}}{3^2d^2{m}^\frac{3}{2}}
\end{align*}
\mysssec{假设现在极精确的测量已经揭示出库仑定律的误差。两个点电荷之间的作用力为$\boldsymbol{F}=\dfrac{q_1q_2}{4\pi\varepsilon_0r^2}\left(1+\frac{r}{\lambda}\right)e^{-\frac{r}{\lambda}}\boldsymbol{e}_r$式中, 入是一个新的自然常数。你的任务是按照这个新发现重新表述静电学。假定叠加原理仍然成立。\\
(a)电场是什么?\\
(b)求出一个点电荷的电势。\\
(c)对一个位于原点的点电荷, 证明$\oiint_S\boldsymbol{E}\cdot\dif\boldsymbol{S}+\frac{1}{\lambda^2}\iiint_VU\dif\tau =\frac{q}{\varepsilon_0}$
}
(a):
\begin{align*}
    \boldsymbol{E}&=\dfrac{q}{4\pi\varepsilon_0r^2}\left(1+\frac{r}{\lambda}\right)e^{-\frac{r}{\lambda}}\boldsymbol{e}_r
\end{align*}

(b):
\begin{align*}
    U&=\int_{R}^{\infty}\dfrac{q}{4\pi\varepsilon_0r^2}\left(1+\frac{r}{\lambda}\right)e^{-\frac{r}{\lambda}}\dif r\\
    &=\dfrac{q}{4\pi\varepsilon_0\lambda}\int_{R}^{\infty}\left(\frac{\lambda^2}{r^2}+\frac{\lambda}{r}\right)e^{-\frac{r}{\lambda}}\dif \frac{r}{\lambda}\\
    &=-\dfrac{q}{4\pi\varepsilon_0\lambda}\int_{R}^{\infty}\left(\frac{\lambda^2}{r^2}--\frac{\lambda}{r}\right)e^{-\frac{r}{\lambda}}\dif -\frac{r}{\lambda}\\
    &\xlongequal{u=-\frac{r}{\lambda}}-\dfrac{q}{4\pi\varepsilon_0\lambda}\int_{-\frac{R}{\lambda}}^{-\infty}\left(\frac{1}{u^2}-\frac{1}{u}\right)e^{u}\dif u\\
    &=-\left.\frac{1}{u}e^{u}\right|_{-\frac{R}{\lambda}}^{-\infty}\dfrac{q}{4\pi\varepsilon_0\lambda}\\
    &=e^{-\frac{R}{\lambda}}\dfrac{q}{4\pi\varepsilon_0R}
\end{align*}

(c):
\begin{align*}
    \oiint_S\boldsymbol{E}\cdot\dif\boldsymbol{S}&=4\pi R^2\dfrac{q}{4\pi\varepsilon_0R^2}\left(1+\frac{R}{\lambda}\right)e^{-\frac{r}{\lambda}}
    =\dfrac{q}{\varepsilon_0}\left(1+\frac{R}{\lambda}\right)e^{-\frac{r}{\lambda}}\\
    \frac{1}{\lambda^2}\iiint_VU\dif\tau&=4\pi\frac{1}{\lambda^2}\int_0^Re^{-\frac{r}{\lambda}}\dfrac{q}{4\pi\varepsilon_0r}r^2\dif r\\
    &=\int_0^R-\frac{r}{\lambda}e^{-\frac{r}{\lambda}}\dfrac{q}{\varepsilon_0}\dif -\frac{r}{\lambda}\\
    &\xlongequal{u=-\frac{r}{\lambda}}\int_0^{-\frac{R}{\lambda}}ue^{u}\dfrac{q}{\varepsilon_0}\dif u\\
    &=\left.\left(u-1\right)e^{u}\dfrac{q}{\varepsilon_0}\right|_0^{-\frac{R}{\lambda}}\\
    &=\left(-\frac{R}{\lambda}-1\right)e^{-\frac{R}{\lambda}}\dfrac{q}{\varepsilon_0}
    +\dfrac{q}{\varepsilon_0}\\
     \frac{q}{\varepsilon_0}&=\oiint_S\boldsymbol{E}\cdot\dif\boldsymbol{S}+\frac{1}{\lambda^2}\iiint_VU\dif\tau
\end{align*}
\mysssec{假设一个电场
$E_1=ax,\, E_2=E_3 =0$
, 电荷密度为什么?}
\begin{align*}
    \varepsilon_0\frac{\partial ax}{\partial x}
    &=a
\end{align*}

\mysssec{所有的静电学特性都是从库仑定律的$\frac{1}{r^2}$以及叠加原理导出的。因此也可以对牛顿万有
引力构建类似的理论。什么是一个半径为$R$, 质量为$M$的球体的引力能?假设质量密度是均匀的。利用所
得结果估计太阳的引力能}

根据库仑定律与万有引力定律的数学相似性, 我们可以通过替换常数直接得到均匀球体的引力自能。

在静电学中, 均匀带电球体的静电能为:
\begin{equation}
    W_e = \frac{3}{5} \frac{1}{4\pi\varepsilon_0} \frac{Q^2}{R}
\end{equation}
利用类比关系得到质量为 $M$ 的均匀球体的引力能:
\begin{equation}
    W_g = -\frac{3}{5} \frac{GM^2}{R}
\end{equation}

假设球体密度为 $\rho = \frac{M}{\frac{4}{3}\pi R^3}$。考虑已形成半径为 $r$ 的球核, 其质量为 $m(r) = \frac{4}{3}\pi \rho r^3$。现从无穷远处移近一层厚度为 $dr$ 的薄壳, 其质量为 $dm = 4\pi \rho r^2 dr$。此过程引力所做的功为:
\begin{align}
    dW &= -\frac{G m(r) dm}{r} \\
       &= -\frac{G}{r} \left( \frac{4}{3}\pi \rho r^3 \right) \left( 4\pi \rho r^2 dr \right) \\
       &= -\frac{16}{3} \pi^2 G \rho^2 r^4 dr
\end{align}
对整个球体从 $0$ 到 $R$ 积分:
\begin{align}
    W_g &= \int_{0}^{R} -\frac{16}{3} \pi^2 G \rho^2 r^4 dr \\
        &= -\frac{16}{3} \pi^2 G \rho^2 \left[ \frac{1}{5} r^5 \right]_{0}^{R} \\
        &= -\frac{16}{15} \pi^2 G \rho^2 R^5
\end{align}
将 $\rho^2 = \frac{M^2}{\frac{16}{9}\pi^2 R^6}$ 代入上式, 化简得:
\begin{equation}
    W_g = -\frac{3}{5} \frac{GM^2}{R}
\end{equation}

取太阳参数:$M \approx {2.0e30}{kg}$, $R \approx {7.0e8}{m}$, 引力常数 $G \approx {6.67e-11}{m^3.kg^{-1}.s^{-2}}$。代入公式得:
\begin{align*}
    W &= -\frac{3}{5} \frac{6.67 \times 10^{-11} \times (2.0 \times 10^{30})^2}{7.0 \times 10^8} \\
            &\approx -{2.28e41}{J}
\end{align*}
\mysssec{我们知道导体上的电荷是分布于其表面的, 但是在表面上是如何分布的不是很容易确定
的。电荷面密度可以直接计算的著名例子是椭圆面:
$\dfrac{x_1^2}{r_1^2}+\dfrac{x_2^2}{r_2^2}+\dfrac{x_3^2}{r_3^2}=1$
对这种情况
$\sigma=\dfrac{Q}{4\pi r_1r_2r_3}\left(\dfrac{x_1^2}{r_1^4}+\dfrac{x_2^2}{r_2^4}+\dfrac{x_3^2}{r_3^4}\right)^{-\frac{1}{2}}$\\
(a)一个半径为$R$的圆盘的净电荷面密度
$\sigma$;\\
(b)位于$x,y$平面一条无限长的导体“丝带”的电荷面密度$\sigma$, 丝带沿$y$轴放置, 宽度$2a$;\\
(c)求出一个从$x_3=-a$到$x_3=a$的导体“针”每单位长度的电荷。}
(a):令$r_1=r_2$
\begin{align*}
    \sigma&=\dfrac{Q}{4\pi r_1^2r_3}\left(\dfrac{x_1^2+x_2^2}{r_1^4}+\dfrac{x_3^2}{r_3^4}\right)^{-\frac{1}{2}}\\
    &=\dfrac{Q}{4\pi r_1^2r_3}\left(\dfrac{x_1^2+x_2^2}{r_1^4}+\dfrac{r_1^2-{x_1^2-x_2^2}}{r_3^2r_1^2}\right)^{-\frac{1}{2}}\\
    &=\dfrac{Q}{4\pi r_1^2}\left(\dfrac{x_1^2+x_2^2}{r_1^4}r_3^2+\dfrac{r_1^2-{x_1^2-x_2^2}}{r_1^2}\right)^{-\frac{1}{2}}\\
    &\xlongequal{r_3\rightarrow 0}\dfrac{Q}{4\pi r_1^2}\left(\dfrac{r_1^2-{x_1^2-x_2^2}}{r_1^2}\right)^{-\frac{1}{2}}
\end{align*}

(b):令$r_2=\infty$
\begin{align*}
    \sigma&=\dfrac{\lambda}{4\pi r_1r_3}\left(\dfrac{x_1^2}{r_1^4}+\dfrac{x_3^2}{r_3^4}\right)^{-\frac{1}{2}}\\
    &=\dfrac{\lambda}{2\pi \sqrt{r_1^2-x_1^2}}
\end{align*}

(c):
\begin{align*}
    \sigma&=\dfrac{Q}{4\pi r_1^2}\left(\dfrac{x_1^2+x_2^2}{r_1^4}r_3^2+\dfrac{r_1^2-{x_1^2-x_2^2}}{r_1^2}\right)^{-\frac{1}{2}}\\
    &=\dfrac{Q}{4\pi r_1^2}\left(\dfrac{r_1^2r_3^2+r_1^2x_3^2}{r_1^4}+\dfrac{x_3^2}{r_1^2r_3^2}\right)^{-\frac{1}{2}}\\
    &\xlongequal{r_1\rightarrow 0}\dfrac{Q}{2\pi \sqrt{a^2-x_3^2}}
\end{align*}
