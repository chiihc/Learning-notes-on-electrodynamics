\cleardoublepage

\section{物质中的电场}
\subsection{极化}
\subsubsection{电介质}
诱导偶极矩:
\begin{align*}
    \boldsymbol{d}=\alpha\boldsymbol{E}
\end{align*}
$\alpha$称为原子极化率

考虑一个简单的原子模型:  
原子核位于原点,带电量 $+q$;  
电子云为半径为 $a$ 的均匀带电球体,总电荷为 $-q$。

在无外场时,体系整体电中性,电偶极矩为零。

现施加一弱、均匀的外电场$E_0$
假设外场足够弱,使电子云仅发生微小整体位移而保持形状不变。


\[
\boldsymbol E_{0}
=
\frac{\boldsymbol d}{4\pi\varepsilon_0 a^3}
\]

\[
\boldsymbol d
=
q\,\boldsymbol s
=
4\pi\varepsilon_0 a^3\,\boldsymbol E_0.
\]


\begin{equation}{
\alpha = 4\pi\varepsilon_0 a^3.
}
\end{equation}