\cleardoublepage

\section{物质中的电场}
\subsection{极化}
\subsubsection{诱导偶极子}
诱导偶极矩:
\begin{align*}
    \boldsymbol{d}=\alpha\boldsymbol{E}
\end{align*}
$\alpha$称为原子极化率

考虑一个简单的原子模型:  
原子核位于原点,带电量 $+q$;  
电子云为半径为 $a$ 的均匀带电球体,总电荷为 $-q$。

在无外场时,体系整体电中性,电偶极矩为零。

现施加一弱、均匀的外电场$E_0$
假设外场足够弱,使电子云仅发生微小整体位移而保持形状不变。


\[
\boldsymbol E_{0}
=
\frac{\boldsymbol d}{4\pi\varepsilon_0 a^3}
\]

\[
\boldsymbol d
=
q\boldsymbol s
=
4\pi\varepsilon_0 a^3\boldsymbol E_0
\]


\begin{equation}{
\alpha = 4\pi\varepsilon_0 a^3
}
\end{equation}
\subsubsection{偶极子在电场}
偶极子在电场中的力矩:
\begin{equation}
    \boldsymbol{N}=\boldsymbol{d}\times\boldsymbol{E}
\end{equation}

偶极子在电场中的能量:
\begin{align*}
    U&=\int \rho U_{ext}\dif \tau\\
    &=-\int \boldsymbol{d}\cdot\boldsymbol{\nabla}\delta(\boldsymbol{r}) U_{ext}\dif \tau\\
    &=\int \boldsymbol{d}\cdot\delta(\boldsymbol{r}) \boldsymbol{\nabla}U_{ext}\dif \tau\\
    &=-\boldsymbol{d}\cdot\boldsymbol{E}
\end{align*}

偶极子在电场中的力:
\begin{align*}
    \boldsymbol{F}&=-\boldsymbol{\nabla}U\\
    &=\boldsymbol{\nabla}(\boldsymbol{d}\cdot\boldsymbol{E})\\
    &=(\boldsymbol{d}\cdot\boldsymbol{\nabla})\boldsymbol{E}
    +(\boldsymbol{E}\cdot\boldsymbol{\nabla})\boldsymbol{d}\\
    &=(\boldsymbol{d}\cdot\boldsymbol{\nabla})\boldsymbol{E}
\end{align*}

\mysssec{一个氢原子(其玻尔半径为$a$)位于两个相距$l$的金属板中间,估计在这个装置中电离原子所需的电压为多少。}
\begin{align*}
    qs
&=
4\pi\varepsilon_0 a^3 E_0\\
    U
&=
\frac{4\pi\varepsilon_0 a^3l}{qs} 
\end{align*}
\mysssec{基态氢原子电子云其电荷密度为$\rho=\dfrac{e}{\pi a^3}\mathrm{e}^{-\frac{2r}{a}}$, 求极化率}
\begin{align*}
    E&=\frac{1}{4\pi\varepsilon_0s^2}\int_{0}^{s}4\pi r^2\dfrac{e}{\pi a^3}\mathrm{e}^{-\frac{2r}{a}}\dif r\\
    &=\frac{1}{4\pi\varepsilon_0s^2}\int_{0}^{s} -\frac{4r^2}{a^2}\dfrac{e}{2 }\mathrm{e}^{-\frac{2r}{a}}\dif -\frac{2r}{a}\\
    &=-\frac{1}{4\pi\varepsilon_0s^2}\int_{0}^{-\frac{2s}{a}} \dfrac{e}{2 }u^2\mathrm{e}^u\dif u\\
    &=-\frac{1}{4\pi\varepsilon_0s^2}\left. \dfrac{e}{2 }(u^2-2u+2)\mathrm{e}^u\right|_{0}^{-\frac{2s}{a}}\\
    &=-\frac{1}{4\pi\varepsilon_0s^2}\dfrac{e}{2}(\frac{4s^2}{a^2}+\frac{4s}{a}+2)\mathrm{e}^{-\frac{2s}{a}}+\frac{1}{4\pi\varepsilon_0s^2}{e}\\
    &\xlongequal{s<<a}\frac{1}{4\pi\varepsilon_0s^2}\dfrac{e}{2}(\frac{4s^2}{a^2}+\frac{4s}{a}+2)\frac{2s}{a}+\frac{e}{4\pi\varepsilon_0s^2}\\
    &=\frac{ea}{2\pi\varepsilon_0s}\\
    \alpha&=\frac{es2\pi\varepsilon_0s}{ea}=\frac{2\pi \varepsilon_0s^2}{a}
\end{align*}
\mysssec{证明相距为$\boldsymbol{r}$的两个偶极子的相互作用能为$U=\dfrac{\boldsymbol{d}_1\cdot\boldsymbol{d}_2-3(\boldsymbol{d}_1\cdot\boldsymbol{e}_r)(\boldsymbol{d}_2\cdot\boldsymbol{e}_r)}{4\pi\varepsilon_0r^3}$}
\begin{align*}
    U&=-\boldsymbol{d}_1\cdot\boldsymbol{E}_2\\
    &=-\boldsymbol{d}_1\cdot\dfrac{3(\boldsymbol{d}_2\cdot\boldsymbol{e}_r)\boldsymbol{e}_r-\boldsymbol{d}_2}{4\pi\varepsilon_0r^3}\\
    &=\dfrac{\boldsymbol{d}_1\cdot\boldsymbol{d}_2-3(\boldsymbol{d}_1\cdot\boldsymbol{e}_r)(\boldsymbol{d}_2\cdot\boldsymbol{e}_r)}{4\pi\varepsilon_0r^3}\\
\end{align*}
\mysssec{一个偶极子$\boldsymbol{d}$与一个点电荷$q$相距为,$\boldsymbol{d}$与$\boldsymbol{r}$的夹角为$\theta$。\\
(a)作用在$\boldsymbol{d}$上的力为多少?\\
(b)作用在$q$上的力为多少?}
\begin{align*}
    \boldsymbol{F}&=(\boldsymbol{d}\cdot\boldsymbol{\nabla})\boldsymbol{E}\\
    &=(d_i\frac{\partial}{\partial x_i})E_j\boldsymbol{e}_j\\
    &=(d_i\frac{\partial}{\partial x_i})\frac{qx_j}{4\pi\varepsilon_0r^3}\boldsymbol{e}_j\\
    &=d_i\frac{q}{4\pi\varepsilon_0r^3}\boldsymbol{e}_j\frac{\partial x_j}{\partial x_i}
    +(d_ix_j\frac{\partial}{\partial x_i})\frac{q}{4\pi\varepsilon_0r^3}\boldsymbol{e}_j\\
    &=d_i\frac{q}{4\pi\varepsilon_0r^3}\boldsymbol{e}_j\delta_{ij}
    -d_ix_j\frac{qx_i}{\pi\varepsilon_0r^4}\boldsymbol{e}_j\\
    &=d_i\frac{q}{4\pi\varepsilon_0r^3}\boldsymbol{e}_i
    -\boldsymbol{d}\cdot\boldsymbol{x}x_j\frac{q}{\pi\varepsilon_0r^4}\boldsymbol{e}_j\\
    &=\boldsymbol{d}\frac{q}{4\pi\varepsilon_0r^3}
    -\boldsymbol{d}\cdot\boldsymbol{x}\boldsymbol{x}\frac{q}{\pi\varepsilon_0r^4}
\end{align*}
\begin{align*}
    \boldsymbol{F}&=q\boldsymbol{E}\\
    &=q\dfrac{3(\boldsymbol{d}\cdot\boldsymbol{e}_r)\boldsymbol{e}_r-\boldsymbol{d}}{4\pi\varepsilon_0r^3}
\end{align*}
\subsection{极化物体的电场}
\begin{equation}
    U=\frac{1}{4\pi\varepsilon_0}
\left(
\iiint_V \frac{\rho_b(\boldsymbol r')}{|\boldsymbol r-\boldsymbol r'|}\dif\tau'
+
\oiint_S \frac{\sigma_b(\boldsymbol r')}{|\boldsymbol r-\boldsymbol r'|}\dif S'
\right)
\end{equation}
表面电荷:
\begin{equation}
    \sigma_b=\boldsymbol{P}\cdot\boldsymbol{\hat{n}}
\end{equation}
内部电荷:
\begin{equation}
    \rho_b=-\boldsymbol{\nabla}\cdot\boldsymbol{P}
\end{equation}
\mysssec{一个半径为$R$的球的极化强度矢量为
$\boldsymbol P=k\boldsymbol r$\\
(a)计算束缚电荷$\sigma_b$和$\rho_b$。\\
(b)求出球内部和外部的电场。}
(a):
\begin{align*}
    \sigma_b&=\boldsymbol{P}\cdot\boldsymbol{\hat{n}}\\
    &=kR\\
    \rho_b&=-\boldsymbol{\nabla}\cdot\boldsymbol{P}\\
    &=-3k
\end{align*}

(b):
\begin{align*}
    \boldsymbol{E}&=\frac{\int_{0}^{r}4\pi kr'^3\dif r'}{4\pi r^2}\boldsymbol{e}_r\\
    &=\frac{ kr^2}{4}\boldsymbol{e}_r
\end{align*}

球内体束缚电荷与表面束缚电荷的总量为
\[
Q
=
\int_V \rho_b\,\mathrm dV
+
\oint_S \sigma_b\,\mathrm dS
=
(-3k)\frac{4\pi R^3}{3}
+
kR\cdot 4\pi R^2
=0.
\]