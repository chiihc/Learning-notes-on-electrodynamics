\cleardoublepage

\section{物质中的电场}
\subsection{极化}
\subsubsection{诱导偶极子}
诱导偶极矩:
\begin{align*}
    \boldsymbol{d}=\alpha\boldsymbol{E}
\end{align*}
$\alpha$称为原子极化率

考虑一个简单的原子模型:  
原子核位于原点, 带电量 $+q$;  
电子云为半径为 $a$ 的均匀带电球体, 总电荷为 $-q$。

在无外场时, 体系整体电中性, 电偶极矩为零。

现施加一弱、均匀的外电场$E_0$
假设外场足够弱, 使电子云仅发生微小整体位移而保持形状不变。


\[
\boldsymbol E_{0}
=
\frac{\boldsymbol d}{4\pi\varepsilon_0 a^3}
\]

\[
\boldsymbol d
=
q\boldsymbol s
=
4\pi\varepsilon_0 a^3\boldsymbol E_0
\]


\begin{equation}{
\alpha = 4\pi\varepsilon_0 a^3
}
\end{equation}
\subsubsection{偶极子在电场}
偶极子在电场中的力矩:
\begin{equation}
    \boldsymbol{N}=\boldsymbol{d}\times\boldsymbol{E}
\end{equation}

偶极子在电场中的能量:
\begin{align*}
    U&=\int \rho U_{ext}\dif \tau\\
    &=-\int \boldsymbol{d}\cdot\boldsymbol{\nabla}\delta(\boldsymbol{r}) U_{ext}\dif \tau\\
    &=\int \boldsymbol{d}\cdot\delta(\boldsymbol{r}) \boldsymbol{\nabla}U_{ext}\dif \tau\\
    &=-\boldsymbol{d}\cdot\boldsymbol{E}
\end{align*}

偶极子在电场中的力:
\begin{align*}
    \boldsymbol{F}&=-\boldsymbol{\nabla}U\\
    &=\boldsymbol{\nabla}(\boldsymbol{d}\cdot\boldsymbol{E})\\
    &=(\boldsymbol{d}\cdot\boldsymbol{\nabla})\boldsymbol{E}
    +(\boldsymbol{E}\cdot\boldsymbol{\nabla})\boldsymbol{d}\\
    &=(\boldsymbol{d}\cdot\boldsymbol{\nabla})\boldsymbol{E}
\end{align*}

\mysssec{一个氢原子(其玻尔半径为$a$)位于两个相距$l$的金属板中间, 估计在这个装置中电离原子所需的电压为多少。}
\begin{align*}
    qs
&=
4\pi\varepsilon_0 a^3 E_0\\
    U
&=
\frac{4\pi\varepsilon_0 a^3l}{qs} 
\end{align*}
\mysssec{基态氢原子电子云其电荷密度为$\rho=\dfrac{e}{\pi a^3}\mathrm{e}^{-\frac{2r}{a}}$, 求极化率}
\begin{align*}
    E&=\frac{1}{4\pi\varepsilon_0s^2}\int_{0}^{s}4\pi r^2\dfrac{e}{\pi a^3}\mathrm{e}^{-\frac{2r}{a}}\dif r\\
    &=\frac{1}{4\pi\varepsilon_0s^2}\int_{0}^{s} -\frac{4r^2}{a^2}\dfrac{e}{2 }\mathrm{e}^{-\frac{2r}{a}}\dif -\frac{2r}{a}\\
    &=-\frac{1}{4\pi\varepsilon_0s^2}\int_{0}^{-\frac{2s}{a}} \dfrac{e}{2 }u^2\mathrm{e}^u\dif u\\
    &=-\frac{1}{4\pi\varepsilon_0s^2}\left. \dfrac{e}{2 }(u^2-2u+2)\mathrm{e}^u\right|_{0}^{-\frac{2s}{a}}\\
    &=-\frac{1}{4\pi\varepsilon_0s^2}\dfrac{e}{2}(\frac{4s^2}{a^2}+\frac{4s}{a}+2)\mathrm{e}^{-\frac{2s}{a}}+\frac{1}{4\pi\varepsilon_0s^2}{e}\\
    &\xlongequal{s<<a}\frac{1}{4\pi\varepsilon_0s^2}\dfrac{e}{2}(\frac{4s^2}{a^2}+\frac{4s}{a}+2)\frac{2s}{a}+\frac{e}{4\pi\varepsilon_0s^2}\\
    &=\frac{ea}{2\pi\varepsilon_0s}\\
    \alpha&=\frac{es2\pi\varepsilon_0s}{ea}=\frac{2\pi \varepsilon_0s^2}{a}
\end{align*}
\mysssec{证明相距为$\boldsymbol{r}$的两个偶极子的相互作用能为$U=\dfrac{\boldsymbol{d}_1\cdot\boldsymbol{d}_2-3(\boldsymbol{d}_1\cdot\boldsymbol{e}_r)(\boldsymbol{d}_2\cdot\boldsymbol{e}_r)}{4\pi\varepsilon_0r^3}$}
\begin{align*}
    U&=-\boldsymbol{d}_1\cdot\boldsymbol{E}_2\\
    &=-\boldsymbol{d}_1\cdot\dfrac{3(\boldsymbol{d}_2\cdot\boldsymbol{e}_r)\boldsymbol{e}_r-\boldsymbol{d}_2}{4\pi\varepsilon_0r^3}\\
    &=\dfrac{\boldsymbol{d}_1\cdot\boldsymbol{d}_2-3(\boldsymbol{d}_1\cdot\boldsymbol{e}_r)(\boldsymbol{d}_2\cdot\boldsymbol{e}_r)}{4\pi\varepsilon_0r^3}\\
\end{align*}
\mysssec{一个偶极子$\boldsymbol{d}$与一个点电荷$q$相距为, $\boldsymbol{d}$与$\boldsymbol{r}$的夹角为$\theta$。\\
(a)作用在$\boldsymbol{d}$上的力为多少?\\
(b)作用在$q$上的力为多少?}
\begin{align*}
    \boldsymbol{F}&=(\boldsymbol{d}\cdot\boldsymbol{\nabla})\boldsymbol{E}\\
    &=(d_i\frac{\partial}{\partial x_i})E_j\boldsymbol{e}_j\\
    &=(d_i\frac{\partial}{\partial x_i})\frac{qx_j}{4\pi\varepsilon_0r^3}\boldsymbol{e}_j\\
    &=d_i\frac{q}{4\pi\varepsilon_0r^3}\boldsymbol{e}_j\frac{\partial x_j}{\partial x_i}
    +(d_ix_j\frac{\partial}{\partial x_i})\frac{q}{4\pi\varepsilon_0r^3}\boldsymbol{e}_j\\
    &=d_i\frac{q}{4\pi\varepsilon_0r^3}\boldsymbol{e}_j\delta_{ij}
    -d_ix_j\frac{qx_i}{\pi\varepsilon_0r^4}\boldsymbol{e}_j\\
    &=d_i\frac{q}{4\pi\varepsilon_0r^3}\boldsymbol{e}_i
    -\boldsymbol{d}\cdot\boldsymbol{x}x_j\frac{q}{\pi\varepsilon_0r^4}\boldsymbol{e}_j\\
    &=\boldsymbol{d}\frac{q}{4\pi\varepsilon_0r^3}
    -\boldsymbol{d}\cdot\boldsymbol{x}\boldsymbol{x}\frac{q}{\pi\varepsilon_0r^4}
\end{align*}
\begin{align*}
    \boldsymbol{F}&=q\boldsymbol{E}\\
    &=q\dfrac{3(\boldsymbol{d}\cdot\boldsymbol{e}_r)\boldsymbol{e}_r-\boldsymbol{d}}{4\pi\varepsilon_0r^3}
\end{align*}
\subsection{极化物体的电场}
\begin{equation}
    U=\frac{1}{4\pi\varepsilon_0}
\left(
\iiint_V \frac{\rho_b(\boldsymbol r')}{|\boldsymbol r-\boldsymbol r'|}\dif\tau'
+
\oiint_S \frac{\sigma_b(\boldsymbol r')}{|\boldsymbol r-\boldsymbol r'|}\dif S'
\right)
\end{equation}
表面电荷:
\begin{equation}
    \sigma_b=\boldsymbol{P}\cdot\boldsymbol{\hat{n}}
\end{equation}
内部电荷:
\begin{equation}
    \rho_b=-\boldsymbol{\nabla}\cdot\boldsymbol{P}
\end{equation}
\mysssec{一个半径为$R$的球的极化强度矢量为
$\boldsymbol P=k\boldsymbol r$\\
(a)计算束缚电荷$\sigma_b$和$\rho_b$。\\
(b)求出球内部和外部的电场。}
(a):
\begin{align*}
    \sigma_b&=\boldsymbol{P}\cdot\boldsymbol{\hat{n}}\\
    &=kR\\
    \rho_b&=-\boldsymbol{\nabla}\cdot\boldsymbol{P}\\
    &=-3k
\end{align*}

(b):
\begin{align*}
    \boldsymbol{E}&=\frac{\int_{0}^{r}4\pi kr'^3\dif r'}{4\pi r^2}\boldsymbol{e}_r\\
    &=\frac{ kr^2}{4}\boldsymbol{e}_r
\end{align*}

球内体束缚电荷与表面束缚电荷的总量为
\[
Q
=
\int_V \rho_b\,\mathrm dV
+
\oint_S \sigma_b\,\mathrm dS
=
(-3k)\frac{4\pi R^3}{3}
+
kR\cdot 4\pi R^2
=0.
\]

\mysssec{一个圆柱体, 半径为$a$, 长度为$L$, 具有均匀极化强度$\boldsymbol{P}$, 方向垂直圆柱面轴线\\
(a)求电场强度\\
(b)证明$\boldsymbol{E}=\dfrac{a^2}{2\varepsilon_0s^2}[2(\boldsymbol{P}\cdot\boldsymbol{e}_s)\boldsymbol{e}_s-\boldsymbol{P}]$}

(a):边界条件
\begin{align*}
    \sigma_b=\boldsymbol{P}\cdot\boldsymbol{\hat{n}}=P\cos \phi
\end{align*}

球内
\begin{align*}
    U&=\sum_{m=0}^{\infty}D_ms^{m}\cos m\phi+D_m's^{-m}\cos m\phi\\
    -\varepsilon_0\frac{\partial U}{\partial s}
    &=\sum_{m=0}^{\infty}\varepsilon_0mD_ms^{m-1}\cos m\phi
    -\varepsilon_0mD_m's^{-m-1}\cos m\phi\\
    \sigma_b
    &=\sum_{m=0}^{\infty}\varepsilon_0mD_ms^{m-1}\cos m\phi
    -\varepsilon_0mD_m's^{-m-1}\cos m\phi\\
    P\cos \phi
    &=\sum_{m=0}^{\infty}\varepsilon_0mD_ms^{m-1}\cos m\phi
    -\varepsilon_0mD_m's^{-m-1}\cos m\phi\\
    P&=\varepsilon_0D_1
    -\varepsilon_0D_1's^{-2}
\end{align*}

球外
\begin{align*}
    D_1'&=-\frac{a^2P}{2\varepsilon_0}
\end{align*}

(b):
\begin{align*}
    U&=-\frac{a^2P}{2\varepsilon_0}s^{-1}\cos\phi\\
    \boldsymbol{E}&=-\boldsymbol{\nabla}U\\
    \boldsymbol{E}&=\dfrac{\partial U}{\partial s}\boldsymbol{s}+\dfrac{1}{s}\dfrac{\partial U}{\partial \phi}\boldsymbol{\phi}\\
    \boldsymbol{E}&=-\dfrac{\partial \frac{a^2P}{2\varepsilon_0}s^{-1}\cos\phi}{\partial s}\boldsymbol{s}
    -\dfrac{1}{s}\dfrac{\partial \frac{a^2P}{2\varepsilon_0}s^{-1}\cos\phi}{\partial \phi}\boldsymbol{\phi}\\
    \boldsymbol{E}&=\frac{a^2P}{2\varepsilon_0}s^{-2}\cos\phi\boldsymbol{s}
    +\dfrac{1}{s}\frac{a^2P}{2\varepsilon_0}s^{-1}\sin\phi\boldsymbol{\phi}
\end{align*}
\begin{align*}
    \dfrac{a^2}{2\varepsilon_0s^2}[2(\boldsymbol{P}\cdot\boldsymbol{e}_s)\boldsymbol{e}_s-\boldsymbol{P}]
    &=\dfrac{a^2}{2\varepsilon_0s^2}(2P_s\boldsymbol{e}_s-P_s\boldsymbol{e}_s-P_\phi\boldsymbol{e}_\phi)\\
    &=\dfrac{a^2}{2\varepsilon_0s^2}(P_s\boldsymbol{e}_s-P_\phi\boldsymbol{e}_\phi)
\end{align*}
\subsection{电位移矢量}
\begin{equation}
    \boldsymbol{D}\equiv\varepsilon_0\boldsymbol{E}+\boldsymbol{P}
\end{equation}
\begin{equation}
    \boldsymbol{\nabla}\cdot\boldsymbol{D}=\rho
\end{equation}
\begin{equation}
    \boldsymbol{\nabla}\times\boldsymbol{D}=\boldsymbol{P}
\end{equation}
\subsubsection{边界条件}
\begin{align}
    D_{\text{上}}^\bot -D_{\text{下}}^\bot&=\sigma_f\\
D_{\text{上}}^{\parallel} -D_{\text{下}}^{\parallel}&=P_{\text{上}}^{\parallel} -P_{\text{下}}^{\parallel}\\
E_{\text{上}}^\bot -E_{\text{下}}^\bot&=\frac{\sigma}{\varepsilon_0}\\
E_{\text{上}}^{\parallel} -E_{\text{下}}^{\parallel}&=0
\end{align}
\subsection{线性电介质}
\subsubsection{电极化率,介电常数,相对介电常数和能量}
\begin{equation}
    \chi_e\equiv\frac{P}{\varepsilon_0 E}
\end{equation}
\begin{equation}
    \varepsilon_r\equiv1+\chi_e=\frac{\varepsilon}{\varepsilon_0}\Rightarrow \boldsymbol{D}=\varepsilon\boldsymbol{E}
\end{equation}
\begin{equation}
W=\frac{\varepsilon_0}{2}\iiint \boldsymbol{D}\cdot\boldsymbol{E}\dif\tau
\end{equation}
\mysssec{一个线性均匀的球形电介质材料放置于一个均匀的外电场$E_0$中, 求球内部的电场强度}
边界条件:
\begin{align*}
    &U_\text{内}=U_\text{外},&r=R\\
    &\varepsilon\frac{\partial U_\text{内}}{\partial r}=\varepsilon_0\frac{\partial U_\text{外}}{\partial r},&r=R\\
    &U_\text{外}\to-{E}_0r\cos\theta,&r\to\infty
\end{align*}
\begin{align*}
    U_\text{内}&
=
A_l r^l
P_{l}(\cos\theta)\\
    U_\text{外}&
=
B_l r^{-(l+1)}
P_{l}(\cos\theta)-E_0r\cos\theta\\
A_l R^l
P_{l}(\cos\theta)&=B_l R^{-(l+1)}P_{l}(\cos\theta)-E_0R\cos\theta\\
\varepsilon\frac{\partial U_\text{内}}{\partial r}&=\varepsilon_0\frac{\partial U_\text{外}}{\partial r}\\
\varepsilon\frac{\partial A_l r^l
P_{l}(\cos\theta)}{\partial r}&=\varepsilon_0\frac{\partial B_l r^{-(l+1)}
P_{l}(\cos\theta)-E_0r\cos\theta}{\partial r}\\
\varepsilon lA_l R^{l-1}
P_{l}(\cos\theta)&=-\varepsilon_0B_l (l+1)R^{-(l+2)}
P_{l}(\cos\theta)-E_0\cos\theta
\end{align*}

解得
\begin{align*}
    A_1&=-\frac{3}{\varepsilon_r+2}E_0\\
    B_1&=\frac{\varepsilon_r-1}{\varepsilon_r+2}E_0
\end{align*}
\mysssec{在一个半径为$a$的不带电导体球外面覆盖一层绝缘外壳, 相对介电常数为$\varepsilon_r$,外壳半径为$b$, 放置在均匀外场$E_0$中, 求绝缘外壳内部的电场}
边界条件:
\begin{align*}
    &U_{a\text{内}}=U_{a\text{外}},&r=a\\
    &U_{b\text{内}}=U_{b\text{外}},&r=b\\
    &\varepsilon\frac{\partial U_{a\text{内}}}{\partial r}-\varepsilon_0\frac{\partial U_{a\text{外}}}{\partial r}=\sigma_f,&r=a\\
    &U_\text{外}\to-{E}_0r\cos\theta,&r\to\infty
\end{align*}

由静电屏蔽并设球心处电势为零可得
\begin{align*}
        U_{a\text{内}}&
=0\\
U_{a\text{外}b\text{内}}&=
A_{l2} r^l
P_{l}(\cos\theta)
+B_{l2} r^{-l-1}
P_{l}(\cos\theta)\\
U_{b\text{外}}&=
B_{l3} r^{-l-1}
P_{l}(\cos\theta)-E_0r\cos\theta
\end{align*}
\begin{align*}
U_{a\text{内}}&=U_{a\text{外}}\\
0&=A_{l2}a^l
P_{l}(\cos\theta)
+B_{l2}a^{-l-1}
P_{l}(\cos\theta)\\
A_{l2}+B_{l2}a^{-2l-1}&=0
\end{align*}
\begin{align*}
U_{b\text{内}}&=U_{b\text{外}}\\
A_{l2}b^l
P_{l}(\cos\theta)
+B_{l2}b^{-l-1}
P_{l}(\cos\theta)&=
B_{l3}b^{-l-1}
P_{l}(\cos\theta)-E_0b\cos\theta\\
A_{l2}b^l
+B_{l2}b^{-l-1}&=
B_{l3}b^{-l-1},\,(l\neq  1)
\end{align*}
\begin{align*}
% \varepsilon\frac{\partial U_{a\text{内}}}{\partial r}&=\varepsilon_0\frac{\partial U_{a\text{外}}}{\partial r}\\
% 0&=\varepsilon_0\frac{\partial A_{l2} r\cos\theta
% +B_{12} r^{-1}
% \cos\theta}{\partial r}\\
% 0&=\varepsilon_0 A_{l2}
% -B_{12} a^{-2}\\
\varepsilon\frac{\partial U_{b\text{内}}}{\partial r}&=\varepsilon_0\frac{\partial U_{b\text{外}}}{\partial r}\\
\varepsilon\frac{\partial A_{l2} r^l
P_{l}(\cos\theta)
+B_{l2} r^{-l-1}
P_{l}(\cos\theta)}{\partial r}&=\varepsilon_0\frac{\partial 
B_{l3} r^{-l-1}
P_{l}(\cos\theta)-E_0r\cos\theta}{\partial r}\\
\varepsilon lA_{l2} b^{l-1}
P_{l}(\cos\theta)
-\varepsilon (l+1)B_{l2} b^{-l-2}
P_{l}(\cos\theta)&=-\varepsilon_0(l+1)
B_{l3} b^{-l-2}
P_{l}(\cos\theta)-\varepsilon_0E_0\cos\theta\\
\varepsilon lA_{l2} b^{l-1}
-\varepsilon (l+1)B_{l2} b^{-l-2}&=-\varepsilon_0(l+1)
B_{l3} b^{-l-2},\,(l\neq  1)
\end{align*}

\begin{align*}
    &\left\{\begin{aligned}
        A_{l2}+B_{l2}a^{-2l-1}&=0\\
A_{l2}b^l
+B_{l2}b^{-l-1}&=
B_{l3}b^{-l-1}\\
\varepsilon lA_{l2} b^{l-1}
-\varepsilon (l+1)B_{l2} b^{-l-2}&=-\varepsilon_0(l+1)
B_{l3} b^{-l-2}
\end{aligned}\right.,\,(l\neq  1)
\end{align*}

由于线性无关, 所以$A_{l2}=B_{l2}=B_{l3}=0$
\begin{align*}&\left\{
\begin{aligned}
        A_{12}+B_{12}a^{-3}&=0\\
A_{12}b
+B_{12}b^{-2}&=
B_{13}b^{-2}-E_0b\\
\varepsilon A_{12}
-\varepsilon 2B_{12} b^{-3}&=-\varepsilon_02
B_{13} b^{-3}-\varepsilon_0E_0
\end{aligned}
\right.\\
&\left\{
\begin{aligned}
A_{12}+a^{-3}B_{12}&=0\\
bA_{12}
+b^{-2}B_{12}
-b^{-2}B_{13}&=
-E_0b\\
\varepsilon A_{12}
-\varepsilon 2b^{-3}B_{12} 
+\varepsilon_02b^{-3}
B_{13} &=-\varepsilon_0E_0
\end{aligned}
\right.\\
&\left\{
\begin{aligned}
-a^{-3}B_{12}b
+b^{-2}B_{12}
-b^{-2}B_{13}&=
-E_0b\\
-\varepsilon a^{-3}B_{12}
-\varepsilon 2b^{-3}B_{12} 
+\varepsilon_02b^{-3}
B_{13} &=-\varepsilon_0E_0
\end{aligned}
\right.
\\
&\left\{
\begin{aligned}
-\varepsilon_0a^{-3}B_{12}
+\varepsilon_0b^{-3}B_{12}
-\varepsilon_0b^{-3}B_{13}&=
-\varepsilon_0E_0\\
-\varepsilon a^{-3}B_{12}
-\varepsilon 2b^{-3}B_{12} 
+\varepsilon_02b^{-3}
B_{13} &=-\varepsilon_0E_0
\end{aligned}
\right.\\
&\left\{
\begin{aligned}
-\varepsilon_0a^{-3}B_{12}
+\varepsilon_0b^{-3}B_{12}
-\varepsilon_0b^{-3}B_{13}&=
-\varepsilon_0E_0\\
-(\varepsilon+2\varepsilon_0) a^{-3}B_{12}
-(\varepsilon-2\varepsilon_0) 2b^{-3}B_{12} 
 &=-3\varepsilon_0E_0
\end{aligned}
\right.
\end{align*}
\mysssec{一个半径为$a$, 带电荷为$Q$的导体球被一个半径为$b$的线性电介质包裹, 求总能量}
\begin{align*}
    W&=\int_{a}^{b}4\pi r\varepsilon_0\left(\frac{Q}{4\pi\varepsilon r^2}\right)^2\dif R
+\int_{b}^{\infty}4\pi r\varepsilon_0\left(\frac{Q}{4\pi\varepsilon_0r^2}\right)^2\dif R=\left(\frac{1}{a^4}-\frac{1}{b^4}\right)\frac{\varepsilon_0Q^2}{16\pi\varepsilon^2}+\frac{1}{b^4}\frac{Q^2}{16\pi\varepsilon_0}
\end{align*}
\mysssec{内径为$a$外径为$b$的同轴圆柱导体管竖直放置在充满密度为$\rho$的油性电介质的桶中, 内部的金属管电势恒为$U_0$, 外管接地, 求两管之间油上升的高度$h$}
\begin{align*}
    W&=\iiint\frac{D^2\pi(b^2-a^2)}{2\varepsilon}\dif z\dif S+\int_{0}^{h}\rho g\pi(b^2-a^2)z\dif z\\
    W&=\iiint\frac{D^2\pi(b^2-a^2)}{2\varepsilon}\dif z\dif S+\frac{1}{2}\rho g\pi(b^2-a^2)h^2
    \end{align*}
    \begin{align*}
    \frac{\partial W}{\partial h}&=0\\
    \frac{\partial \iiint\dfrac{D^2\pi(b^2-a^2)}{2\varepsilon}\dif z\dif S+\dfrac{1}{2}\rho g\pi(b^2-a^2)h^2}{\partial h}&=0\\
    \frac{\partial -\iint\dfrac{D^2\pi(b^2-a^2)\chi_eh}{2\varepsilon}\dif S+\dfrac{1}{2}\rho g\pi(b^2-a^2)h^2}{\partial h}&=0\\
    -\iint\dfrac{D^2\pi(b^2-a^2)\chi_e}{2\varepsilon}\dif S+\rho g\pi(b^2-a^2)h&=0\\
    -\int_a^br\left(\frac{U_0\varepsilon}{r\ln b-r\ln a}\right)^2\dfrac{\chi_e}{2\varepsilon}\dif r+\rho gh&=0\\
    \rho gh&=\frac{U_0^2\varepsilon}{\ln b-\ln a}\dfrac{\chi_e}{2}\\
    h&=\frac{U_0^2\varepsilon\chi_e}{2\rho g\ln b-2\rho g\ln a}
\end{align*}
\subsection{习题}
\mysssec{电荷$q$位于一个线性均匀电介质球的中心, 求球内电场强度, 极化强度和体束缚电荷}
\begin{align*}
    D&=\frac{q}{4\pi r^2}\\
    E&=\frac{D}{\varepsilon}=\frac{q}{4\pi r^2\varepsilon}\\
    P&=D-\varepsilon_0E=\frac{q\chi_e}{4\pi r^2\varepsilon_r}\\
    \rho_b&=-\boldsymbol{\nabla}\cdot\boldsymbol{P}=0
\end{align*}
\mysssec{在两个不同介电常数电介质表面, 电场线会发生弯折, 如果在边界没有自由电荷, 证明:\\$\dfrac{\tan\theta_1}{\tan\theta_2}=\dfrac{\varepsilon_1}{\varepsilon_2}$}
\begin{align}
    D_{1}^\bot&=D_{2}^\bot\\
    \varepsilon_1E_1^\bot&=\varepsilon_2E_2^\bot\\
E_1^{\parallel}&=E_2^{\parallel}\\
\dfrac{\tan\theta_1}{\tan\theta_2}
&=\dfrac{\dfrac{E_2^\bot}{E_2^{\parallel}}}{\dfrac{E_1^\bot}{E_1^{\parallel}}}\\
&=\frac{E_1^{\parallel}E_2^\bot}{E_1^\bot E_2^{\parallel}}\\
&=\frac{E_2^\bot}{E_1^\bot}\\
&=\dfrac{\varepsilon_1}{\varepsilon_2}
\end{align}
\mysssec{一个点电偶极子$d$镶嵌在一个线性均匀介电球(电极化
率为$\chi_e$, 半径为$R$)的中心。求在球体内部和外部的电势能}
边界条件:
\begin{align*}
    &U_{\text{内}}=U_{\text{外}},&r=R\\
    &\varepsilon\frac{\partial U_{\text{内}}}{\partial r}=\varepsilon_0\frac{\partial U_{\text{外}}}{\partial r},&r=R\\
    &U_\text{外}\to0,&r\to\infty\\
    &U_{\text{内}}\to\frac{d\cos\theta}{4\pi\varepsilon_0r^2},&r\to0
\end{align*}
\begin{align*}
U_{\text{内}}&=
A_l r^l
P_l(\cos\theta)
+\frac{d\cos\theta}{4\pi\varepsilon_0r^2}\\
U_{b\text{外}}&=
B_l r^{-l-1}
P_l(\cos\theta)\\
A_l R^l
P_l(\cos\theta)
+\frac{d\cos\theta}{4\pi\varepsilon_0R^2}&=
B_l R^{-l-1}
P_l(\cos\theta)\\
A_l R^l&=
B_l R^{-l-1},\,(l\neq1)\\
\varepsilon\frac{\partial A_l r^l
P_l(\cos\theta)
+\dfrac{d\cos\theta}{4\pi\varepsilon_0r^2}}{\partial r}&=\varepsilon_0\frac{\partial B_l r^{-l-1}
P_l(\cos\theta)}{\partial r}\\
\varepsilon lA_l r^{l-1}P_l(\cos\theta)
- \frac{2\varepsilon d\cos\theta}{4\pi\varepsilon_0r^3}&=-(l+1)\varepsilon_0B_l r^{-l-2}
P_l(\cos\theta)\\
\varepsilon lA_l R^{l-1}&=
-(l+1)\varepsilon_0B_l R^{-l-2},\,(l\neq1)
\end{align*}

由于线性无关, 所以$A_{l}=B_{l}=0$
\begin{align*}
    A_1 R
P_1(\cos\theta)
+\frac{d\cos\theta}{4\pi\varepsilon_0R^2}&=
B_1' R^{-2}
P_l(\cos\theta)\\
A_1 R
+\frac{d}{4\pi\varepsilon_0R^2}&=
B_1 R^{-2}\\
\varepsilon A_1 P_1(\cos\theta)
- \frac{2\varepsilon d\cos\theta}{4\pi\varepsilon_0R^3}&=-2\varepsilon_0B_1 R^{-3}
P_1(\cos\theta)\\
\varepsilon A_1
- \frac{\varepsilon d}{2\pi\varepsilon_0R^3}&=-2\varepsilon_0B_1 R^{-3}\\
B_1&=\frac{3d\varepsilon}{4\pi\varepsilon_0(\varepsilon+2\varepsilon_0)}\\
A_1&=\frac{2d(\varepsilon-\varepsilon_0)}{4\pi\varepsilon_0(\varepsilon+2\varepsilon_0)R^3}
\end{align*}

\newpage