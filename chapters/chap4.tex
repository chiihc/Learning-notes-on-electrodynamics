\cleardoublepage

\section{静磁学}
\subsection{洛伦兹力定律}
\subsubsection{磁力}
\begin{equation}
    \boldsymbol{F}=\boldsymbol{v}\times\boldsymbol{B}
\end{equation}
\begin{equation}
    \boldsymbol{F}=I\int\dif\boldsymbol{l}\times\boldsymbol{B}
\end{equation}
\begin{equation}
    \boldsymbol{F}=\int\boldsymbol{J}\times\boldsymbol{B}\dif\tau
\end{equation}
\subsubsection{电流}
\begin{equation}
    \boldsymbol{\nabla}\cdot\boldsymbol{J}=-\frac{\partial \rho}{\partial t}
\end{equation}
% \mysssec{假设在某个区域磁场形式为
% $\boldsymbol B =kz\boldsymbol{\hat{x}}$。一个边长为$a$的方形线框, 处在$yz$平面内, 中心在原点, 如果通逆时针的电流。求作用在线框的力。}
% \begin{align*}
%     \boldsymbol{F}=I\int\dif\boldsymbol{l}\times\boldsymbol{B}=0
% \end{align*}
\mysssec{电流$I$沿半径为$a$的导线流动, \\
(a)如果电流均匀分布在导线表面上, 那么面电流密度$K$为多少?\\
(b)如果体电流密度分布反比于到中心轴的距离, 那么$J$为什么?}
(a):$\frac{I}{2\pi a}$

(b):
\begin{align*}
    I&=2\pi\int_{0}^{a}\frac{k}{r}r\dif r=2\pi ak\to k=\frac{I}{2\pi a}\\
    J&=\frac{I}{2\pi ar}
\end{align*}
\mysssec{(a)一个留声机唱片表面有均匀的电荷面密度$\sigma$。如果它以角速度$\omega$旋转, 那么离中心距离为$r$处的面
电流密度$K$为多少?\\
(b)电荷$Q$均匀分布在半径为$R$的固体球内, 中心在原点, 并以恒定角速度$\omega$绕轴旋转。求出球内任
意点($r,\theta,\phi$)处的电流密度$J$。}
(a):$K=\sigma v=\sigma \omega r$

(b):$J=\rho v=\frac{4Q}{3\pi R^3}r\sin\phi\omega$
\subsection{比奥萨伐尔定律}
\subsubsection{稳恒电流}
\begin{equation}
    \boldsymbol{\nabla}\cdot\boldsymbol{J}=0
\end{equation}
\subsubsection{稳恒电流的磁场}
\begin{equation}
    \boldsymbol{B}=\frac{\mu_0}{4\pi}I\int\frac{\dif \boldsymbol{l}\times\boldsymbol{r}}{r^3}
\end{equation}
\begin{equation}
    \boldsymbol{\nabla}\cdot\boldsymbol{B}=0
\end{equation}
\begin{equation}
    \boldsymbol{\nabla}\times\boldsymbol{B}=\mu_0\boldsymbol{J}
\end{equation}
\mysssec{求出通有稳恒电流$I$的$n$边规则多边形线框中心的磁场, 同样$R$为中心到边的距离。}
\begin{align*}
    B=2n\frac{\mu_0 I}{4\pi R}\sin\frac{2\pi}{2n}=\frac{n\mu_0 I}{2\pi R}\sin\frac{\pi}{n}
\end{align*}
% \mysssec{电流$I$沿半径为$a$的导线流动, 求磁场\\
% (a)如果电流均匀分布在导线表面上\\
% (b)如果体电流密度分布反比于到中心轴的距离}
% (a):$2\pi aB=\mu_0 I\to B=\dfrac{\mu_0 I}{2\pi aB}$

% (b):
\mysssec{一个大平板的厚度从$z=-a$到$z=a$, 它的电流密度为$\boldsymbol{J}=J\boldsymbol{\hat x}$, 
求出平板内外的磁场。}
\begin{align*}
    B=\left\{\begin{aligned}
        &aJ&,|z|>a\\
        &zJ&,|z|<=a\\
    \end{aligned}\right.
\end{align*}
\mysssec{一个大平行板电容器, 上极板有均匀电荷$\sigma$, 下极板有均匀电荷$-\sigma$, 
两极板以恒定速度$v$运动\\
(a)求出两极板间的磁场\\
(b)求出作用在上极板单位面积上的磁力及其方向\\
(c)为了使磁力和电场力平衡, 速度应为多大}
(a):$
    B=\mu_0J=\mu_0\sigma v$

(b):$F=\sigma vB=\mu_0\sigma^2 v^2$

(c):$\mu_0\sigma^2 v^2=\dfrac{\sigma}{2\varepsilon_0}\to v^2=\dfrac{1}{2\mu_0\sigma \varepsilon_0}$
\subsection{磁矢势}


在静磁学中,磁场满足
$
\boldsymbol{\nabla}\cdot\boldsymbol{B}=0.
$
这意味着磁场是一个\emph{无源场}。

根据向量分析中的 Helmholtz 定理:

\begin{quote}
在单连通区域内,任何散度为零的光滑矢量场,都可以表示为某个矢量场的旋度。
\end{quote}

因此存在矢量场
$\boldsymbol{A}(\boldsymbol{r})$,使得
\begin{equation}
\boxed{
\boldsymbol{B}=\boldsymbol{\nabla}\times\boldsymbol{A}.
}
\end{equation}

该矢量场 $\boldsymbol{A}$ 称为{磁矢势}。



磁矢势并非唯一。
若 $\boldsymbol{A}$ 给出磁场 $\boldsymbol{B}$,则对任意标量场
$\Lambda(\boldsymbol{r})$,定义
\begin{equation}
\boldsymbol{A}'=\boldsymbol{A}+\boldsymbol{\nabla}\Lambda,
\end{equation}
有
\begin{equation}
\boldsymbol{\nabla}\times\boldsymbol{A}'
=
\boldsymbol{\nabla}\times\boldsymbol{A}
=
\boldsymbol{B}.
\end{equation}

磁场 $\boldsymbol{B}$ 是可观测量,而磁矢势 $\boldsymbol{A}$ 本身不是。
不同的 $\boldsymbol{A}$ 描述的是同一个物理磁场。


为了固定这种不唯一性,通常施加附加条件。
最常用的是{库仑规范}:
\begin{equation}
{
\boldsymbol{\nabla}\cdot\boldsymbol{A}=0.
}
\end{equation}


静磁场满足安培定律:
\begin{equation}
\boldsymbol{\nabla}\times\boldsymbol{B}=\mu_0\boldsymbol{J}.
\end{equation}

代入 $\boldsymbol{B}=\boldsymbol{\nabla}\times\boldsymbol{A}$,得到
\begin{equation}
\boldsymbol{\nabla}\times(\boldsymbol{\nabla}\times\boldsymbol{A})
=
\mu_0\boldsymbol{J}.
\end{equation}

利用恒等式
\[
\boldsymbol{\nabla}\times(\boldsymbol{\nabla}\times\boldsymbol{A})
=
\boldsymbol{\nabla}(\boldsymbol{\nabla}\cdot\boldsymbol{A})
-
\nabla^2\boldsymbol{A},
\]
并在库仑规范下 $\boldsymbol{\nabla}\cdot\boldsymbol{A}=0$,得
\begin{equation}
{
\nabla^2\boldsymbol{A}=-\mu_0\boldsymbol{J}.
}
\end{equation}

这是一组三个{形式上独立的泊松方程}。


在无穷远处 $\boldsymbol{A}\to0$ 的条件下,上式的解为
\begin{equation}
{
\boldsymbol{A}(\boldsymbol{r})
=
\frac{\mu_0}{4\pi}
\int
\frac{\boldsymbol{J}(\boldsymbol{r}')}{|\boldsymbol{r}-\boldsymbol{r}'|}
\,\dif^3 r'.
}
\end{equation}

该表达式在形式上与静电学中电势的解完全类似。

\subsubsection{含表面电流时径向磁矢势的边界条件}
对麦克斯韦方程在一条跨越界面的无穷小曲面上积分,并取极限 $\epsilon\to0$,得到边界条件
\begin{equation}
\boldsymbol{B}_{||\text{out}}-\boldsymbol{B}_{||\text{in}}=\mu_0\boldsymbol{K}.
\end{equation}

代入 $\boldsymbol{B}=\boldsymbol\nabla\times\boldsymbol{A}$,得到边界条件
\begin{equation}
\boldsymbol\nabla\times\boldsymbol{A}_{||\text{out}}-
\boldsymbol\nabla\times\boldsymbol{A}_{||\text{in}}=\mu_0\boldsymbol{K}
\end{equation}

同理易得
\begin{equation}
    B_{\bot\text{out}}=B_{\bot\text{in}}
\end{equation}

\mysssec{一半径为$R$的球壳,表面带有均匀电荷,电荷面密度为$\sigma$,当它以恒定角速度$\omega$旋转时,求矢势}
边界条件:
\begin{align*}
    &\boldsymbol{A}_{\text{内}}=\boldsymbol{A}_{\text{外}},&r=R\\
    &\frac{\partial A_{\theta\text{外}}}{\partial r}-\frac{\partial A_{\theta\text{内}}}{\partial r}=-\mu_0\sigma\omega R\sin\theta,&r=R\\
    &\boldsymbol{A}_\text{外}\to0,&r\to\infty\\
    &\boldsymbol{A}_{\text{内}}\to\frac{d\cos\theta}{4\pi\varepsilon_0r^2},&r\to0
\end{align*}
\begin{align*}
\boldsymbol{\nabla}^2\boldsymbol{A}_\theta
&=\frac{1}{r^2}\frac{\partial}{\partial r}\left(r^2\dfrac{{\partial A_\theta}}{\partial r}\right)\boldsymbol{e}_\theta
-\frac{1}{r^2\sin^2\theta}A_\theta\boldsymbol{e}_\theta
+\frac{1}{r^2\sin\theta}\left(
\cos\theta\dfrac{{\partial A_\theta}}{\partial \theta}
+\sin\theta\dfrac{{\partial^2 A_\theta}}{\partial \theta^2}\right)\boldsymbol{e}_\theta\\
A_{\theta\text{内}}
&=B_1rP_1^1(\cos\theta)=-B_1\sin\theta r\\
A_{\theta\text{外}}
&=C_1r^{-2}P_1^1(\cos\theta)
=-C_1\sin\theta r^{-2}
\end{align*}
\begin{align*}
\boldsymbol{A}_{\text{内}}=&\boldsymbol{A}_{\text{外}}\\
-B_1\sin\theta R
=&-C_1\sin\theta R^{-2}\\
B_1
=&C_1R^{-3}\\
\frac{\partial A_{\theta\text{外}}}{\partial r}-\frac{\partial A_{\theta\text{内}}}{\partial r}=&-\mu_0\sigma\omega R\sin\theta\\
\frac{\partial -B_1\sin\theta r}{\partial r}-\frac{\partial -C_1\sin\theta r^{-2}}{\partial r}=&-\mu_0\sigma\omega R\sin\theta\\
-B_1\sin\theta  -2C_1\sin\theta R^{-3}=&-\mu_0\sigma\omega R\sin\theta\\
C_1R^{-3}+2C_1R^{-3}=&\mu_0\sigma\omega R\\
C_1=&\frac{1}{3}\mu_0\sigma\omega R^4\\
B_1=&\frac{1}{3}\mu_0\sigma\omega R
\end{align*}
\mysssec{一段长$l$直导线通有电流$I$,它所产生的磁矢势为多少?}
\begin{align*}
    A=\frac{\mu_0}{4\pi}\int_0^l\frac{\dif z'}{\sqrt{s^2+(z-z')^2}}=\frac{\mu_0}{4\pi}\ln\left[\frac{z+\sqrt{s^2+z^2}}{z-l+\sqrt{s^2+(z-l)^2}}\right]\end{align*}
\mysssec{用柱坐标表示,电流密度为多大才能产生磁矢势$\boldsymbol A=k\boldsymbol{e}_\phi$}
\begin{align*}
    \boldsymbol A&=k\boldsymbol{e}_\phi\\
    \boldsymbol\nabla^2\boldsymbol A&=k\boldsymbol\nabla^2\boldsymbol{e}_\phi\\
    -\frac{\mu_0}{k}\boldsymbol{J}
    &=-
\frac{1}{s^2}\boldsymbol{e}_\phi\\
\boldsymbol{J}
    &=
\frac{k}{\mu_0s^2}\boldsymbol{e}_\phi
\end{align*}
\subsection{磁偶极矩}
\begin{equation}
    \boldsymbol{A}_\text{偶极}=\frac{\mu_0}{4\pi}\frac{\boldsymbol{m}\times\boldsymbol{r}}{r^3}
\end{equation}
式中, $\boldsymbol{m}$是磁偶极矩
\begin{equation}
    \boldsymbol{m}\equiv I\boldsymbol{a}
\end{equation}
\mysssec{一个半径为$R$的留声机唱片,带有均匀的面电荷$\sigma$,以恒定角速度$\omega$旋转,求出它的磁偶极矩}
\begin{align*}
    {m}= I{a}=\int_{0}^{R}2\pi\sigma r\omega \pi r^2\dif r
    =\int_{0}^{R}2\pi^2\sigma \omega \pi r^3\dif r
    =\frac{1}{2}\pi^2\sigma \omega \pi R^4
\end{align*}
\subsection{习题}
\mysssec{既然平行电流相互吸引,单根导线中的电流将收缩到沿着中心轴线的一条细束上。然而在实际中,电流一般将均匀地分布于导线中,如果正电荷(密度$\rho_+$)是静止的,负电荷(密度$\rho_-$)以速度$v$运动,证明$\rho_-=\rho_+\gamma^2$,其中$\gamma=1/\sqrt{1-(v/c)^2}$。如果导线整体上是中性的,那么补偿电荷处在什么地方?}
设导线中负电荷的分布为$\rho_-(r)$
\begin{align*}
    W&=\int_{0}^{R}\frac{\varepsilon_0}{2}E^2+\frac{B^2}{2\mu_0}\dif R\\
    W&=\int_{0}^{R}\frac{\varepsilon_0}{2}\left(\int_0^r\frac{\rho_+-\rho_-}{2\pi \varepsilon_0r'}\dif r'\right)^2+\frac{1}{2\mu_0}\left(\int_0^r\frac{\mu_0\rho_-}{2\pi r'}\dif r'\right)^2\dif R\\
\end{align*}