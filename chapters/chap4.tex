\cleardoublepage

\section{静磁学}
\subsection{洛伦兹力定律}
\subsubsection{磁力}
\begin{equation}
    \boldsymbol{F}=\boldsymbol{v}\times\boldsymbol{B}
\end{equation}
\begin{equation}
    \boldsymbol{F}=I\int\dif\boldsymbol{l}\times\boldsymbol{B}
\end{equation}
\begin{equation}
    \boldsymbol{F}=\int\boldsymbol{J}\times\boldsymbol{B}\dif\tau
\end{equation}
\subsubsection{电流}
\begin{equation}
    \boldsymbol{\nabla}\cdot\boldsymbol{J}=-\frac{\partial \rho}{\partial t}
\end{equation}
% \mysssec{假设在某个区域磁场形式为
% $\boldsymbol B =kz\boldsymbol{\hat{x}}$。一个边长为$a$的方形线框, 处在$yz$平面内, 中心在原点, 如果通逆时针的电流。求作用在线框的力。}
% \begin{align*}
%     \boldsymbol{F}=I\int\dif\boldsymbol{l}\times\boldsymbol{B}=0
% \end{align*}
\mysssec{电流$I$沿半径为$a$的导线流动, \\
(a)如果电流均匀分布在导线表面上, 那么面电流密度$K$为多少?\\
(b)如果体电流密度分布反比于到中心轴的距离, 那么$J$为什么?}
(a):$\frac{I}{2\pi a}$

(b):
\begin{align*}
    I&=2\pi\int_{0}^{a}\frac{k}{r}r\dif r=2\pi ak\to k=\frac{I}{2\pi a}\\
    J&=\frac{I}{2\pi ar}
\end{align*}
\mysssec{(a)一个留声机唱片表面有均匀的电荷面密度$\sigma$。如果它以角速度$\omega$旋转, 那么离中心距离为$r$处的面
电流密度$K$为多少?\\
(b)电荷$Q$均匀分布在半径为$R$的固体球内, 中心在原点, 并以恒定角速度$\omega$绕轴旋转。求出球内任
意点($r,\theta,\phi$)处的电流密度$J$。}
(a):$K=\sigma v=\sigma \omega r$

(b):$J=\rho v=\frac{4Q}{3\pi R^3}r\sin\phi\omega$
\subsection{比奥萨伐尔定律}
\subsubsection{稳恒电流}
\begin{equation}
    \boldsymbol{\nabla}\cdot\boldsymbol{J}=0
\end{equation}
\subsubsection{稳恒电流的磁场}
\begin{equation}
    \boldsymbol{B}=\frac{\mu_0}{4\pi}I\int\frac{\dif \boldsymbol{l}\times\boldsymbol{r}}{r^3}
\end{equation}
\begin{equation}
    \boldsymbol{\nabla}\cdot\boldsymbol{B}=0
\end{equation}
\begin{equation}
    \boldsymbol{\nabla}\times\boldsymbol{B}=\mu_0\boldsymbol{J}
\end{equation}
\mysssec{求出通有稳恒电流$I$的$n$边规则多边形线框中心的磁场, 同样$R$为中心到边的距离。}
\begin{align*}
    B=2n\frac{\mu_0 I}{4\pi R}\sin\frac{2\pi}{2n}=\frac{n\mu_0 I}{2\pi R}\sin\frac{\pi}{n}
\end{align*}
% \mysssec{电流$I$沿半径为$a$的导线流动, 求磁场\\
% (a)如果电流均匀分布在导线表面上\\
% (b)如果体电流密度分布反比于到中心轴的距离}
% (a):$2\pi aB=\mu_0 I\to B=\dfrac{\mu_0 I}{2\pi aB}$

% (b):
\mysssec{一个大平板的厚度从$z=-a$到$z=a$, 它的电流密度为$\boldsymbol{J}=J\boldsymbol{\hat x}$, 
求出平板内外的磁场。}
\begin{align*}
    B=\left\{\begin{aligned}
        &aJ&,|z|>a\\
        &zJ&,|z|<=a\\
    \end{aligned}\right.
\end{align*}
\mysssec{一个大平行板电容器, 上极板有均匀电荷$\sigma$, 下极板有均匀电荷$-\sigma$, 
两极板以恒定速度$v$运动\\
(a)求出两极板间的磁场\\
(b)求出作用在上极板单位面积上的磁力及其方向\\
(c)为了使磁力和电场力平衡, 速度应为多大}
(a):$
    B=\mu_0J=\mu_0\sigma v$

(b):$F=\sigma vB=\mu_0\sigma^2 v^2$

(c):$\mu_0\sigma^2 v^2=\dfrac{\sigma}{2\varepsilon_0}\to v^2=\dfrac{1}{2\mu_0\sigma \varepsilon_0}$
\subsection{磁矢势}


在静磁学中,磁场满足
$
\boldsymbol{\nabla}\cdot\boldsymbol{B}=0.
$
这意味着磁场是一个\emph{无源场}。

根据向量分析中的 Helmholtz 定理:

\begin{quote}
在单连通区域内,任何散度为零的光滑矢量场,都可以表示为某个矢量场的旋度。
\end{quote}

因此存在矢量场
$\boldsymbol{A}(\boldsymbol{r})$,使得
\begin{equation}
\boxed{
\boldsymbol{B}=\boldsymbol{\nabla}\times\boldsymbol{A}.
}
\end{equation}

该矢量场 $\boldsymbol{A}$ 称为{磁矢势}。



磁矢势并非唯一。
若 $\boldsymbol{A}$ 给出磁场 $\boldsymbol{B}$,则对任意标量场
$\Lambda(\boldsymbol{r})$,定义
\begin{equation}
\boldsymbol{A}'=\boldsymbol{A}+\boldsymbol{\nabla}\Lambda,
\end{equation}
有
\begin{equation}
\boldsymbol{\nabla}\times\boldsymbol{A}'
=
\boldsymbol{\nabla}\times\boldsymbol{A}
=
\boldsymbol{B}.
\end{equation}

磁场 $\boldsymbol{B}$ 是可观测量,而磁矢势 $\boldsymbol{A}$ 本身不是。
不同的 $\boldsymbol{A}$ 描述的是同一个物理磁场。


为了固定这种不唯一性,通常施加附加条件。
最常用的是{库仑规范}:
\begin{equation}
{
\boldsymbol{\nabla}\cdot\boldsymbol{A}=0.
}
\end{equation}


静磁场满足安培定律:
\begin{equation}
\boldsymbol{\nabla}\times\boldsymbol{B}=\mu_0\boldsymbol{J}.
\end{equation}

代入 $\boldsymbol{B}=\boldsymbol{\nabla}\times\boldsymbol{A}$,得到
\begin{equation}
\boldsymbol{\nabla}\times(\boldsymbol{\nabla}\times\boldsymbol{A})
=
\mu_0\boldsymbol{J}.
\end{equation}

利用恒等式
\[
\boldsymbol{\nabla}\times(\boldsymbol{\nabla}\times\boldsymbol{A})
=
\boldsymbol{\nabla}(\boldsymbol{\nabla}\cdot\boldsymbol{A})
-
\nabla^2\boldsymbol{A},
\]
并在库仑规范下 $\boldsymbol{\nabla}\cdot\boldsymbol{A}=0$,得
\begin{equation}
{
\nabla^2\boldsymbol{A}=-\mu_0\boldsymbol{J}.
}
\end{equation}

这是一组三个{形式上独立的泊松方程}。


在无穷远处 $\boldsymbol{A}\to0$ 的条件下,上式的解为
\begin{equation}
{
\boldsymbol{A}(\boldsymbol{r})
=
\frac{\mu_0}{4\pi}
\int
\frac{\boldsymbol{J}(\boldsymbol{r}')}{|\boldsymbol{r}-\boldsymbol{r}'|}
\,\dif^3 r'.
}
\end{equation}

该表达式在形式上与静电学中电势的解完全类似。

\subsubsection{分离变量法}

在没有电流时

\begin{align*}
    \nabla^2\boldsymbol{A}&=0\\
    \frac{1}{r^2}\frac{\partial}{\partial r}\left(r^2\dfrac{{\partial \boldsymbol{A}}}{\partial r}\right)
    +\frac{1}{r^2\sin^2\theta}\dfrac{{\partial^2 \boldsymbol{A}}}{\partial \phi^2}
    +\frac{1}{r^2\sin\theta}\frac{\partial}{\partial \theta}\left(\sin\theta\dfrac{{\partial \boldsymbol{A}}}{\partial \theta}\right)&=0
\end{align*}

$\boldsymbol r$方向


\subsubsection{含表面电流时径向磁矢势的边界条件}
对麦克斯韦方程在一条跨越界面的无穷小曲面上积分,并取极限 $\epsilon\to0$,得到边界条件
\begin{equation}
\boldsymbol{B}_{||\text{out}}-\boldsymbol{B}_{||\text{in}}=\mu_0\boldsymbol{K}.
\end{equation}

代入 $\boldsymbol{B}=\boldsymbol\nabla\times\boldsymbol{A}$,得到边界条件
\begin{equation}
\boldsymbol\nabla\times\boldsymbol{A}_{||\text{out}}-
\boldsymbol\nabla\times\boldsymbol{A}_{||\text{in}}=\mu_0\boldsymbol{K}
\end{equation}


\subsubsection*{直角坐标系下的形式}

设界面为 $z=0$ 平面,表面电流为
$\boldsymbol{K}=K_x\hat{\boldsymbol{x}}+K_y\hat{\boldsymbol{y}}$。

旋度在直角坐标系中为
\begin{equation}
\nabla\times\boldsymbol{A}=
\begin{pmatrix}
\partial_y A_z-\partial_z A_y\
\partial_z A_x-\partial_x A_z\
\partial_x A_y-\partial_y A_x
\end{pmatrix}.
\end{equation}

代入普适条件并逐分量比较,得到
\begin{align}
\left.\frac{\partial A_x}{\partial z}\right|_{||\text{out}}-
\left.\frac{\partial A_x}{\partial z}\right|_{||\text{in}}&=-\mu_0 K_y,\
\left.\frac{\partial A_y}{\partial z}\right|_{||\text{out}}-
\left.\frac{\partial A_y}{\partial z}\right|_{||\text{in}}&=\mu_0 K_x.
\end{align}

\subsubsection*{三、球坐标系下的形式}

设界面为球面 $r=R$。并设电流与$\phi$无关, 则
\begin{equation*}
\nabla\times\boldsymbol{A}=-\frac{1}{r\sin\theta}\left(\frac{\partial a_\theta}{\partial \phi}
    -\frac{\partial\sin\theta  a_{\phi}}{\partial \theta}\right)\boldsymbol{r}
    +\frac{1}{r}\left(\frac{\partial a_r}{\partial \theta}
    -\frac{\partial ra_\theta}{\partial r}\right)\boldsymbol{\phi}
    +\frac{1}{r\sin\theta }\left(\sin\theta\frac{\partial r a_{\phi}}{\partial r}
    -\frac{\partial a_r}{\partial \phi}
    \right)\boldsymbol{\theta}
\end{equation*}

代入普适边界条件,得到
\begin{equation}
\boxed{
\left.\frac{\partial}{\partial r}\bigl(rA_{\varphi,\text{out}}\bigr)\right|*{r=R}-
\left.\frac{\partial}{\partial r}\bigl(rA*{\varphi,\text{in}}\bigr)\right|*{r=R}
=-\mu_0 R K*\varphi
}
\end{equation}

这是旋转带电球壳等问题中使用的标准形式。

\subsubsection*{四、柱坐标系下的形式}

设界面为圆柱面 $\rho=a$,法向 $\hat{\boldsymbol{n}}=\hat{\boldsymbol{\rho}}$。若
\begin{equation}
\boldsymbol{K}=K_z(\varphi,z),\hat{\boldsymbol{z}},
\qquad
\boldsymbol{A}=A_z(\rho),\hat{\boldsymbol{z}},
\end{equation}

则
\begin{equation}
(\nabla\times\boldsymbol{A})_\varphi=-\frac{\partial A_z}{\partial \rho}.
\end{equation}

边界条件化为
\begin{equation}
\boxed{
\left.\frac{\partial A_{z,\text{out}}}{\partial \rho}\right|*{\rho=a}-
\left.\frac{\partial A*{z,\text{in}}}{\partial \rho}\right|_{\rho=a}
=-\mu_0 K_z
}
\end{equation}


\mysssec{一半径为$R$的球壳,表面带有均匀电荷,电荷面密度为$\sigma$,当它以恒定角速度$\omega$旋转时,求出它在产生的矢势}
边界条件:
\begin{align*}
    &\boldsymbol{A}_{\text{内}}=\boldsymbol{A}_{\text{外}},&r=R\\
    &\frac{\partial A_{\theta\text{外}}}{\partial r}-\frac{\partial A_{\theta\text{内}}}{\partial r}=-\mu_0\sigma\omega R\sin\theta,&r=R\\
    &\boldsymbol{A}_\text{外}\to0,&r\to\infty\\
    &\boldsymbol{A}_{\text{内}}\to\frac{d\cos\theta}{4\pi\varepsilon_0r^2},&r\to0
\end{align*}
\begin{align}
\boldsymbol{\nabla}^2\boldsymbol{A}
&=\frac{1}{r^2}\frac{\partial}{\partial r}\left(r^2\dfrac{{\partial \boldsymbol{A}}}{\partial r}\right)
+\frac{1}{r^2\sin^2\theta}\dfrac{{\partial^2 \boldsymbol{A}}}{\partial \phi^2}
+\frac{1}{r^2\sin\theta}\frac{\partial}{\partial \theta}\left(\sin\theta\dfrac{{\partial \boldsymbol{A}}}{\partial \theta}\right)\nonumber \\
\boldsymbol{\nabla}^2\boldsymbol{A}_\theta
&=\frac{1}{r^2}\frac{\partial}{\partial r}\left(r^2\dfrac{{\partial \boldsymbol{A}_\theta}}{\partial r}\right)
+\frac{1}{r^2\sin^2\theta}\dfrac{{\partial^2 \boldsymbol{A}_\theta}}{\partial \phi^2}
+\frac{1}{r^2\sin\theta}\frac{\partial}{\partial \theta}\left(\sin\theta\dfrac{{\partial \boldsymbol{A}_\theta}}{\partial \theta}\right)\nonumber \\
&=\frac{1}{r^2}\frac{\partial}{\partial r}\left(r^2\dfrac{{\partial A_\theta}\boldsymbol{e}_\theta}{\partial r}\right)
+\frac{1}{r^2\sin^2\theta}\dfrac{{\partial^2 A_\theta}\boldsymbol{e}_\theta}{\partial \phi^2}
+\frac{1}{r^2\sin\theta}\frac{\partial}{\partial \theta}\left(\sin\theta\dfrac{{\partial A_\theta}\boldsymbol{e}_\theta}{\partial \theta}\right)\nonumber \\
&=\frac{1}{r^2}\frac{\partial}{\partial r}\left(r^2\dfrac{{\partial A_\theta}}{\partial r}\right)\boldsymbol{e}_\theta
+\frac{1}{r^2\sin^2\theta}\dfrac{{\partial^2 A_\theta}}{\partial \phi^2}\boldsymbol{e}_\theta
+\frac{1}{r^2\sin\theta}\frac{\partial}{\partial \theta}\left(\sin\theta\dfrac{{\partial A_\theta}\boldsymbol{e}_\theta}{\partial \theta}\right)\nonumber \\
\end{align}

\begin{align*}
\boldsymbol{A}_{\text{内}}&
=
\boldsymbol{B}_l r^lP_{l}(\cos\theta)\\
\boldsymbol{A}_{\text{外}}&
=
\boldsymbol{C}_l r^{-(l+1)}P_{l}(\cos\theta)\\
\boldsymbol{A}_{\text{内}}&=\boldsymbol{A}_{\text{外}}\\
\boldsymbol{B}_l R^lP_{l}(\cos\theta)
&=\boldsymbol{C}_l R^{-(l+1)}P_{l}(\cos\theta)\\
\frac{\partial A_{\phi\text{外}}}{\partial r}-\frac{\partial A_{\phi\text{内}}}{\partial r}&=-\mu_0\sigma\omega R\sin\theta\\
\frac{\partial \boldsymbol{B}_l r^lP_{l}(\cos\theta)}{\partial r}-\frac{\partial \boldsymbol{C}_l r^{-(l+1)}P_{l}(\cos\theta)}{\partial r}&=-\mu_0\sigma\omega R\sin\theta\\
l\boldsymbol{B}_l R^{l-1}P_{l}(\cos\theta)+(l+1)\boldsymbol{C}_l R^{-(l+2)}P_{l}(\cos\theta)&=-\mu_0\sigma\omega R\sin\theta\\
l\boldsymbol{B}_l R^lP_{l}(\cos\theta)+(l+1)\boldsymbol{C}_l R^{-(l+1)}P_{l}(\cos\theta)&=-\mu_0\sigma\omega R^2\sin\theta\\
l\boldsymbol{C}_l R^{-(l+1)}P_{l}(\cos\theta)+(l+1)\boldsymbol{C}_l R^{-(l+1)}P_{l}(\cos\theta)&=-\mu_0\sigma\omega R^2\sin\theta\\
\boldsymbol{C}_l R^{-2}P_1(\cos\theta)+2\boldsymbol{C}_l R^{-2}P_1(\cos\theta)&=-\mu_0\sigma\omega R^2\sin\theta\\  
\boldsymbol{C}_l R^{-2}P_1(\cos\theta)+2\boldsymbol{C}_l R^{-2}P_1(\cos\theta)&=-\mu_0\sigma\omega R^2\sin\theta\\  
\end{align*}